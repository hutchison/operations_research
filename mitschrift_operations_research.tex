\documentclass[
%% alle weiteren Papierformat einstellbar:
a4paper, %apaper,
%% keine Seitenzahlen:
%empty,
%% Schriftgröße (12pt, 11pt (Standard)):
11pt,
%% kleinere Überschriften:
%smallheadings,
]
{scrartcl}

% _das_ Mathepaket schlechthin:
\usepackage[
  %% Nummerierung von Gleichungen links:
  leqno,
  %% Ausgabe von Gleichungen linksbündig:
  fleqn,
]{mathtools}
% und dazu noch ein paar Mathesymbole und so:
% (muss vor mathspec geladen werden)
\usepackage{amsmath, amssymb}

\usepackage[]{parskip}

% chemische Formeln
\usepackage{mhchem}

\usepackage{ifxetex}
\ifxetex
  % Um auch schöne Schriftarten auswählen zu können:
  \usepackage[MnSymbol]{mathspec}

  % Wir wollen, dass alle unsere Schriften für TeX und einander angepasst sind:
  \defaultfontfeatures{Mapping=tex-text, Scale=MatchLowercase}
  % Die Hauptschriftart:
  \setmainfont[]{Minion Pro}
  % Die Matheschriftart:
  \setmathfont(Digits,Latin,Greek)[
    Numbers={Lining, Proportional}
  ]{Minion Pro}
  \setmathrm{Minion Pro}
  % Die Schriftart für serifenlose Texte (z.B. Überschriften):
  \setallsansfonts[]{Myriad Pro}
  % Und die Schriftart für nichtproportionale Texte:
  \setallmonofonts[]{Fira Mono}
\fi

% Deutsche Sprache bei Silbentrennung und Datum:
\usepackage[ngerman]{babel}

% St. Mary Road, liefert Symbole für theoretische Informatik (z.B. \lightning):
%\usepackage{stmaryrd}

% nutze die volle Seite als Satzspiegel:
\usepackage[
  % Randbreite sei 1cm (sonst ist sie 1in):
  cm,
  % Kopf- und Fußzeile werden miteinbezogen:
  headings
]{fullpage}
% verbesserte Tabellen
% bietet u.a. die Spaltenmöglichkeit 'm{width}' = zentrierte Spalte mit fester
% Breite
\usepackage{array}
% kann komplexe Linien in Tabellen ziehen:
%\usepackage{hhline}
% mehrseitige Tabellen:
%\usepackage{longtable}
% Tabellen mit gedehnten Spalten:
\usepackage{tabularx}
% Pimpt enumerate auf (optionales Argument liefert Nummerierung):
\usepackage{enumerate}
% Kann descriptions auf die selbe Höhe bringen:
%\usepackage{enumitem}
% Liefert Hyperlinks (\hyperref, \url, \href}
\usepackage{hyperref}
\hypersetup{%
  colorlinks=true,
  linkcolor=black,
  urlcolor=blue,
}
\usepackage{cleveref}
% Farben (wie bei TikZ):
\usepackage[dvipsnames]{xcolor}
\usepackage[]{pifont}
% Ändert den Zeilenabstand:
\usepackage[
  % nur eine Möglichkeit auswählen:
  singlespacing
  %onehalfspacing
  %doublespacing
]{setspace}
% Schönere Tabellen
% dazu gibt's neue Kommandos:
% - \toprule[(Dicke)], \midrule[(Dicke)], \bottomrule[(Dicke)]
% - \addlinespace: Extrahöhe zwischen Zeilen
\usepackage{booktabs}
% Schöne numerische Zitierungen:
%\usepackage{cite}
% Verbessert den Satz von Abbildungsüberschriften:
%\usepackage{caption}
% Ermöglicht durch \begin{linenumbers} Zeilennummern anzuzeigen:
%\usepackage{lineno}
% Ermöglicht Zugriff auf die letzte Seite (z.B. \pageref{LastPage}):
\usepackage{lastpage}
% Quelltext schön setzen:
%\usepackage{listings}
% Logische Beweise:
%\usepackage{bussproofs}
% Unterstreichungen (\uline, \uuline, \sout: durchgestrichen, \uwave):
%\usepackage{ulem}
% Kann alle möglichen Maße ändern
% will man Querformat, dann:
%\usepackage[landscape]{geometry}
% bietet gestrichelte vert. Linien in Tabellen (':')
%\usepackage{arydshln}
% Quelltext schön setzen:
% (verlangt "xelatex -shell-escape"!)
%\usepackage{minted}
% um Bilder einzubinden:
%\usepackage{graphicx}
% um in Tabellen eine Zelle über mehrere Zeilen laufen zu lassen:
%\usepackage{multirow}
% SI-Einheiten mittels \si{}:
\usepackage[mode=text]{siunitx}
\sisetup{%
  output-decimal-marker={,},
  %per-mode=fraction,
  %exponent-product=\cdot,
}
%\DeclareSIUnit\cal{cal}
%\DeclareSIUnit\diopter{dpt}
%\DeclareSIUnit\fahrenheit{F}
% nette Brüche mittels \sfrac{}{}:
\usepackage{xfrac}

% Coole Zeichnungen:
\usepackage{tikz}
\usetikzlibrary{%
  %backgrounds,
  %mindmap,
  %shapes.geometric,
  %shapes.symbols,
  %shapes.misc,
  %shapes.multipart,
  %positioning,
  %fit,
  calc,
  arrows,
  %automata,
  %trees,
  %decorations.pathreplacing,
  %circuits.ee.IEC,
}

% eigens gebaute Lochmarken:
%\usepackage{eso-pic}
%\AddToShipoutPicture*{
  %\put(\LenToUnit{0mm},\LenToUnit{228.5mm})
    %{\rule{\LenToUnit{20pt}}{\LenToUnit{0.5pt}}}
  %\put(\LenToUnit{0mm},\LenToUnit{68.5mm})
    %{\rule{\LenToUnit{20pt}}{\LenToUnit{0.5pt}}}
%}

% Definitionen und Sätze:
\usepackage[]{amsthm}

\usepackage{wrapfig}

\newtheoremstyle{bonny}% schottisch für „ansehnlich“
  {9pt}% measure of space to leave above the theorem. E.g.: 3pt
  {6pt}% measure of space to leave below the theorem. E.g.: 3pt
  {}% name of font to use in the body of the theorem
  {}% measure of space to indent
  {\bfseries}% name of head font
  {\smallskip}% punctuation between head and body
  {\newline}% space after theorem head; " " = normal interword space
  {}% Manually specify head

\theoremstyle{bonny}

\newtheorem{definition}{Definition}
\newtheorem{gesetz}{Gesetz}
\newtheorem{satz}{Satz}
\newtheorem{beispiel}{Beispiel}
\newtheorem{bemerkung}{Bemerkung}

% coole Kopf- und Fußzeilen:
\usepackage{fancyhdr}
% Seitenstil ist natürlich fancy:
\pagestyle{fancy}
% alle Felder löschen:
\fancyhf{}
% Veranstaltung:
% Linie oben/unten:
\renewcommand{\headrulewidth}{0.0pt}
\renewcommand{\footrulewidth}{0.0pt}

\newcommand{\cmark}{\ding{51}}%
\newcommand{\xmark}{\ding{55}}%
\newcommand{\rot}[1]{\textcolor{BrickRed}{#1}}
\newcommand{\gruen}[1]{\textcolor{ForestGreen}{#1}}
\newcommand{\richtig}{\gruen{\text{\cmark}}}
\newcommand{\falsch}{\rot{\text{\xmark}}}


\fancyhead[L]{Operations Research}
\fancyhead[C]{}
%\fancyhead[R]{}
%\fancyfoot[L]{}
%\fancyfoot[C]{\thepage}
\fancyfoot[C]{\thepage\,/\,\pageref{LastPage}}

\newcommand{\dist}{\text{d}}
\newcommand{\vor}{\text{vor}}
\newcommand{\nil}{\text{nil}}

\DeclareSIUnit\EUR{EUR}
\DeclareSIUnit\USD{USD}
\DeclareSIUnit\GBP{GBP}
\DeclareSIUnit\JPY{JPY}
\DeclareSIUnit\RUB{RUB}
\DeclareSIUnit\CHF{CHF}

\begin{document}

\section{Theorie}%
\label{sec:theorie}

Ein lineares Optimierungsproblem oder lineares Programmierungsproblem (LP) ist
ein Optimierugsproblem, eine lineare Funktion (die Zielfunktion) zu optimieren
(maximieren/minimieren) unter Beachtung von linearen Restriktionen.

Das Grundmodell der linearen Optimierung ist
\begin{gather*}
  \max \quad c_1 x_1 + c_2 x_2 + \ldots + c_n x_n\\
  \begin{aligned}
    \textup{unter}
    \quad
    a_{11} x_1 + a_{12} x_2 + \ldots + a_{1n} x_n \leq b_1\\
    a_{21} x_1 + a_{22} x_2 + \ldots + a_{2n} x_n \leq b_2\\
    \vdots\\
    a_{m1} x_1 + a_{m2} x_2 + \ldots + a_{mn} x_n \leq b_m
  \end{aligned}
\end{gather*}
Dabei sind
\begin{description}
  \item[$c_j$] Zielfunktionskoeffizienten (bekannt),
  \item[$x_j$] Strukturvariablen,
  \item[$a_{ij}$] technische Koeffizienten (bekannt),
  \item[$b_i$] Kapazitäten/Restriktionswerte (bekannt),
  \item[$x_j \geq 0$] Nichtnegativitätsbedingungen.
\end{description}
Die bekannten Werte $c_j, a_{ij}, b_i$ bilden zusammen eine Instanz/Eingabe des
LP-Modells/Problems.

In Vektor-Matrix-Notation geschrieben:
\begin{gather*}
  \max \quad c^T x\\
  \begin{aligned}
    \textup{unter}
    \quad
    Ax \leq b\\
    x \geq 0
  \end{aligned}
\end{gather*}

Das Grundmodell stellt keine willkürlichen Einschränkungen dar.
\begin{itemize}
  \item Minimierungsproblem:
    \begin{align*}
      \min z & = - \max (-z)
    \end{align*}

  \item Untere Abschätzung:
    \begin{align*}
      a_{i1} x_1 + a_{i2} x_2 + \ldots + a_{in} x_n \geq b_i
    \end{align*}
    ist äquivalent zu
    \begin{align*}
      -a_{i1} x_1 - a_{i2} x_2 - \ldots - a_{in} x_n \leq -b_i
    \end{align*}

  \item Gleichung:
    \begin{align*}
      a_{i1} x_1 + a_{i2} x_2 + \ldots + a_{in} x_n = b_i
    \end{align*}
    ist äquivalent zu
    \begin{align*}
      a_{i1} x_1 + a_{i2} x_2 + \ldots + a_{in} x_n \leq b_i\\
      -a_{i1} x_1 - a_{i2} x_2 - \ldots - a_{in} x_n \leq -b_i
    \end{align*}

\end{itemize}

\section{Übungen}%
\label{sec:ubungen}

\subsection{Übung 1}%
\label{sub:ubung_1}

\paragraph{Aufgabe 1}%
\label{par:aufgabe_1}
Lösen Sie das folgende LP mit dem Simplex-Algorithmus:
\begin{gather*}
  \max \quad x_1 + 3 x_2 - x_3\\
  \begin{aligned}
    \textup{unter}
    \quad
    2 x_1 + 2 x_2 - x_3 & \leq 10\\
    3 x_1 - 2 x_2 + x_3 & \leq 10\\
    x_1   - 3 x_2 + x_3 & \leq 10\\
    x_1, x_2, x_3       & \geq 0
  \end{aligned}
\end{gather*}

Zuerst führen wir die Schlupfvariablen für die übrigbleibenden Kapazitäten der
einzelnen Nebenbedingungen ein:
\begin{align*}
  \begin{array}{rcccrcrcr}
    x_4 & = & 10 & - & 2 x_1 & - & 2 x_2 & + & x_3\\
    x_5 & = & 10 & - & 3 x_1 & + & 2 x_2 & - & x_3\\
    x_6 & = & 10 & - & x_1   & + & 3 x_2 & - & x_3\\
    \midrule
    z   & = &    &   & x_1   & + & 3 x_2 & - & x_3
  \end{array}
\end{align*}
Die Variable mit dem höchsten Koeffizienten in der Zielfunktion ist $x_2$, also
ist $x_2$ die Pivotspalte. Für die Pivotzeile müssen wir
$\min \left\{
  \frac{\bar{b}_i}{\bar{a}_{ik}} \mid i \in B, \bar{a}_{ik} > 0
  \right\}$
finden. Berechnen wir diese:
\begin{align*}
  \begin{array}{rcccrcrcrcr}
    x_4 & = & 10 & - & 2 x_1 & - & 2 x_2 & + & x_3 & & 2/10 = 5\\
    x_5 & = & 10 & - & 3 x_1 & + & 2 x_2 & - & x_3 & & \rot{-2/10 < 0}\\
    x_6 & = & 10 & - & x_1   & + & 3 x_2 & - & x_3 & & \rot{-3/10 < 0}\\
    \midrule
    z   & = &    &   & x_1   & + & 3 x_2 & - & x_3
  \end{array}
\end{align*}
Damit ist $x_4$ unsere Pivotzeile. Jetzt wird $x_2$ eine neue Basisvariable,
also müssen wir die erste Zeile nach $x_2$ auflösen und in den restlichen Zeilen
$x_2$ ersetzen. Erstmal auflösen:
\begin{align*}
  \begin{array}{rcccrcrcr}
    x_2 & = & 5  & - & x_1   & + & \frac{1}{2} x_3 & - & \frac{1}{2} x_4\\
    x_5 & = & 10 & - & 3 x_1 & + & 2 x_2           & - & x_3\\
    x_6 & = & 10 & - & x_1   & + & 3 x_2           & - & x_3\\
    \midrule
    z   & = &    &   & x_1   & + & 3 x_2           & - & x_3
  \end{array}
\end{align*}
Dann $x_2$ in den restlichen Zeilen einsetzen:
\begin{align*}
  \def\arraystretch{1.5}
  \begin{array}{rcccrcrcr}
    x_2 & = & 5  & - & x_1   & + & \frac{1}{2} x_3 & - & \frac{1}{2} x_4\\
    x_5 & = & 10 & - & 3 x_1 & + & 2 (5 - x_1 + \frac{1}{2} x_3 - \frac{1}{2} x_4) & - & x_3\\
    x_6 & = & 10 & - & x_1   & + & 3 (5 - x_1 + \frac{1}{2} x_3 - \frac{1}{2} x_4) & - & x_3\\\midrule
    z   & = &    &   & x_1   & + & 3 (5 - x_1 + \frac{1}{2} x_3 - \frac{1}{2} x_4) & - & x_3
  \end{array}
\end{align*}
Und vereinfachen:
\begin{align*}
  \def\arraystretch{1.5}
  \begin{array}{rcccrcrcr}
    x_2 & = & 5  & - & x_1   & + & \frac{1}{2} x_3 & - & \frac{1}{2} x_4\\
    x_5 & = & 20 & - & 5 x_1 &   &                 & - & x_4\\
    x_6 & = & 25 & - & 4 x_1 & + & \frac{1}{2} x_3 & - & \frac{3}{2} x_4\\\midrule
    z   & = & 15 & - & 2 x_1 & + & \frac{1}{2} x_3 & - & \frac{3}{2} x_4
  \end{array}
\end{align*}
In der Zielfunktion hat $x_3$ den höchsten Anstieg, also ist es die Pivotspalte.
Wir suchen die Pivotzeile:
\begin{align*}
  \def\arraystretch{1.5}
  \begin{array}{rcccrcrcrcr}
    x_2 & = & 5  & - & x_1   & + & \frac{1}{2} x_3 & - & \frac{1}{2} x_4 & & \rot{5/(-1/2) < 0}\\
    x_5 & = & 20 & - & 5 x_1 &   &                 & - & x_4\\
    x_6 & = & 25 & - & 4 x_1 & + & \frac{1}{2} x_3 & - & \frac{3}{2} x_4 & & \rot{25/(-1/2) < 0}\\
    \midrule
    z   & = & 15 & - & 2 x_1 & + & \frac{1}{2} x_3 & - & \frac{3}{2} x_4
  \end{array}
\end{align*}
Da alle $\bar{a}_{i3} > 0$ sind, ist das LOP unbeschränkt und damit nicht
lösbar.

\paragraph{Aufgabe 2}%
\label{par:aufgabe_2}
Wenn Sie ein LP mit dem Simplex-Algorithmus lösen: woran erkennen Sie, ob das LP
unendlich viele optimale Lösungen hat?

Ermitteln Sie \textit{alle} optimalen Lösungen des folgenden LP mit dem
Simplex-Algorithmus:
\begin{gather*}
  \max \quad 6x_1 + 12x_2 + 4x_3\\
  \begin{aligned}
    \textup{unter}
    \quad
    3x_1 + 6x_2 + 2x_3 & \leq 6\\
    -x_1 + 2x_2        & \leq 2\\
    x_1, x_2, x_3      & \geq 0
  \end{aligned}
\end{gather*}

Schauen wir mal, wie weit wir kommen. Schlupfvariablen:
\begin{align*}
  \begin{array}{rcrcrcrcr}
    x_4 & = & 6 & - & 3 x_1 & - & 6 x_2 & - & 2 x_3\\
    x_5 & = & 2 & + &   x_1 & - & 2 x_2\\
    \midrule
    z   & = &   &   & 6 x_1 & + & 12 x_2 & + & 4 x_3
  \end{array}
\end{align*}
Damit ist $x_2$ die Pivotspalte. Für die Pivotzeile berechnen wir:
\begin{align*}
  \begin{array}{rcrcrcrcrcr}
    x_4 & = & 6 & - & 3 x_1 & - &  6 x_2 & - & 2 x_3 & & 6/6 = 1\\
    x_5 & = & 2 & + &   x_1 & - &  2 x_2 &   &       & & 2/2 = 1\\
    \midrule
    z   & = &   &   & 6 x_1 & + & 12 x_2 & + & 4 x_3
  \end{array}
\end{align*}
Damit können wir sowohl $x_4$ und $x_5$ als Pivotzeile wählen. Machen wir eine
Fallunterscheidung:
\begin{description}
  \item[$x_4$:] Erstmal die Pivotzeile umstellen:
    \def\arraystretch{1.5}
    \begin{align*}
      \begin{array}{rcrcrcrcr}
        x_2      & = & 1 & - & \frac{1}{2} x_1 & - & \frac{1}{3} x_3 & - & \frac{1}{6} x_4\\
        x_5      & = & 2 & + & x_1             & - & 2 x_2\\
        \midrule
        z        & = &   &   & 6 x_1           & + & 12 x_2          & + & 4 x_3
      \end{array}
    \end{align*}

    Dann $x_2$ ersetzen:
    \def\arraystretch{1.5}
    \begin{align*}
      \begin{array}{rcrcrcrcr}
        x_2      & = & 1  & - & \frac{1}{2} x_1 & - & \frac{1}{3} x_3 & - & \frac{1}{6} x_4\\
        x_5      & = &    &   & 2 x_1           & + & \frac{2}{3} x_3 & + & \frac{1}{3} x_4\\
        \midrule
        z        & = & 12 &   &                 &   &                 & - & 2 x_4
      \end{array}
    \end{align*}

    Damit haben wir die erste Lösung schon gefunden: $x_2 = 1$ und $x_5 = 0$.
    Alle anderen sind gleich $0$.
    Genauer gesagt ist unsere erste Lösung
    $(x_1, x_2, x_3, x_4, x_5) = (0, 1, 0, 0, 0)$.

  \item[$x_5$:] Wieder die Pivotzeile (jetzt $x_5$ nach $x_2$) umstellen und
    einsetzen:
    \begin{align*}
      \begin{array}{rcrcrcrcr}
        x_4      & = &    & - & 6 x_1           & - & 2 x_3 & + & 3 x_5\\
        x_2      & = & 1  & + & \frac{1}{2} x_1 &   &       & - & \frac{1}{2} x_5\\
        \midrule
        z        & = & 12 & + & 12 x_1          & + & 4 x_3 & - & 6 x_5
      \end{array}
    \end{align*}

    Die neue Pivotspalte ist jetzt $x_1$. Wenn wir die Pivotzeile wählen wollen,
    dann erhalten wir:
    \begin{align*}
      \begin{array}{rcrcrcrcrcr}
        x_4      & = &    & - & 6 x_1           & - & 2 x_3 & + & 3 x_5           &  & 0/6 = 0\\
        x_2      & = & 1  & + & \frac{1}{2} x_1 &   &       & - & \frac{1}{2} x_5 &  & 1/(-1/2) < 0\\
        \midrule
        z        & = & 12 & + & 12 x_1          & + & 4 x_3 & - & 6 x_5
      \end{array}
    \end{align*}
    Hier ist keine weitere Verbesserung möglich und wir bekommen die Lösung, die
    wir auch schon vorher hatten:
    $(x_1, x_2, x_3, x_4, x_5) = (0, 1, 0, 0, 0)$.
\end{description}

Die einzige Möglichkeit für weitere Lösungen wäre, wenn wir weiter
Basisvariablen tauschen, auch wenn sich dadurch am Maximalwert der Zielfunktion
nichts mehr ändert. Schauen wir mal was passiert, wenn wir die Darstellung von
$x_5$ weiterrechnen.
\def\arraystretch{1.5}
\begin{align*}
  \begin{array}{rcrcrcrcrcr}
    x_4      & = &    & - & 6 x_1           & - & 2 x_3 & + & 3 x_5           &  & 0/6 = 0\\
    x_2      & = & 1  & + & \frac{1}{2} x_1 &   &       & - & \frac{1}{2} x_5 &  & 1/(-1/2) < 0\\
    \midrule
    z        & = & 12 & + & 12 x_1          & + & 4 x_3 & - & 6 x_5
  \end{array}
\end{align*}
Tauschen wir $x_4$ gegen $x_1$. Umstellen und einsetzen liefert:
\begin{align*}
  \begin{array}{rcrcrcrcrcr}
    x_1      & = &    & - & \frac{1}{3} x_3 & - & \frac{1}{6} x_4  & + & \frac{1}{2} x_5\\
    x_2      & = & 1  & - & \frac{1}{6} x_3 & - & \frac{1}{12} x_4 & - & \frac{1}{4} x_5\\
    \midrule
    z        & = & 12 &   &                 & - & 2 x_4
  \end{array}
\end{align*}
Na das hat nix gebracht. Tolle Wurst. Kann ja auch nix bringen.

Nehmen wir mal den letzten Stand vom Fall $x_4$ her:
\begin{align*}
  \begin{array}{rcrcrcrcr}
    x_2      & = & 1  & - & \frac{1}{2} x_1 & - & \frac{1}{3} x_3 & - & \frac{1}{6} x_4\\
    x_5      & = &    &   & 2 x_1           & + & \frac{2}{3} x_2 & + & \frac{1}{3} x_4\\
    \midrule
    z        & = & 12 &   &                 &   &                 & - & 2 x_4
  \end{array}
\end{align*}
Und tauschen hier $x_2$ durch $x_1$. Umstellen und einsetzen liefert:
\begin{align*}
  \begin{array}{rcrcrcrcr}
    x_1 & = & 2 & - & 2 x_2            & - & \frac{2}{3} x_3 & - & \frac{1}{3} x_4\\
    x_5 & = & 4 & - & \frac{10}{3} x_2 & - & \frac{4}{3} x_3 & - & \frac{1}{3} x_4\\
    \midrule
    z        & = & 12 &   &       &   &                 & - & 2 x_4
  \end{array}
\end{align*}
Damit hätten wir $(x_1, x_2, x_3, x_4, x_5) = (2, 0, 0, 0, 4)$.

Für die nächste Lösung tauschen wir $x_5$ gegen $x_3$ und erhalten:
\begin{align*}
  \begin{array}{rcrcrcrcr}
    x_1      & = &    & - & \frac{1}{3} x_2 & - & \frac{1}{6} x_4 & + & \frac{1}{2} x_5\\
    x_3      & = & 3  & - & \frac{5}{2} x_2 & - & \frac{1}{4} x_4 & - & \frac{3}{4} x_5\\
    \midrule
    z        & = & 12 &   &                 &   &                 & - & 2 x_4
  \end{array}
\end{align*}
Das liefert uns die Lösung $(x_1, x_2, x_3, x_4, x_5) = (0, 0, 3, 0, 0)$.

Und das soll auch erstmal reichen. Es gibt also unendlich viele Lösungen, die
folgende Form haben:
\begin{align*}
  \left\{
    \lambda_1
    \begin{pmatrix}
      2\\
      0\\
      0
    \end{pmatrix}
    +
    \lambda_2
    \begin{pmatrix}
      0\\
      1\\
      0
    \end{pmatrix}
    +
    \lambda_3
    \begin{pmatrix}
      0\\
      0\\
      3
    \end{pmatrix}
    \mid
    \lambda_1 + \lambda_2 + \lambda_3 = 1
  \right\}
\end{align*}

\paragraph{Aufgabe 3}%
\label{par:aufgabe_3}
Lösen Sie folgende LPs mit der 2-Phasen-Methode. Deuten Sie graphisch den
fortschreitenden Lösungsgang an:
\begin{equation}
  \begin{gathered}
    \max \quad 2x_1 + x_2\\
    \begin{aligned}
      \textup{unter}
      \quad
      x_1 + x_2  & \geq 8\\
      3x_1 + x_2 & \geq 12\\
      x_1 + x_2  & \leq 10\\
      x_1, x_2   & \geq 0
    \end{aligned}
  \end{gathered}
\end{equation}

\definecolor{zzttqq}{rgb}{0.6,0.2,0.}
\definecolor{uuuuuu}{rgb}{0.26666666666666666,0.26666666666666666,0.26666666666666666}
\definecolor{qqqqff}{rgb}{0.,0.,1.}
\definecolor{cqcqcq}{rgb}{0.7529411764705882,0.7529411764705882,0.7529411764705882}
\definecolor{red}{rgb}{0.8,0.0,0.0}
\begin{center}
  \begin{tikzpicture}[line cap=round,line join=round,>=triangle 45,x=1.0cm,y=1.0cm]
    \draw [color=cqcqcq,, xstep=1.0cm,ystep=1.0cm] (-1.,-1.) grid (11.,10.);
    \draw[->,color=black] (-1.,0.) -- (11.,0.);
    \foreach \x in {1,2,3,4,5,6,7,8,9,10}
    \draw[shift={(\x,0)},color=black] (0pt,2pt) -- (0pt,-2pt) node[below] {\footnotesize $\x$};
    \draw[->,color=black] (0.,-1.) -- (0.,10.);
    \foreach \y in {1,2,3,4,5,6,7,8,9}
    \draw[shift={(0,\y)},color=black] (2pt,0pt) -- (-2pt,0pt) node[left] {\footnotesize $\y$};
    \draw[color=black] (0pt,-10pt) node[right] {\footnotesize $0$};
    \clip(-1.,-1.) rectangle (11.,10.);
    \fill[color=zzttqq,fill=zzttqq,fill opacity=0.10000000149011612] (2.,6.) -- (1.,9.) -- (10.,0.) -- (8.,0.) -- cycle;
    \draw [color=qqqqff,domain=-1.:11.] plot(\x,{(--8.-1.*\x)/1.});
    \draw [color=qqqqff,domain=-1.:11.] plot(\x,{(--12.-3.*\x)/1.});
    \draw [color=qqqqff,domain=-1.:11.] plot(\x,{(--10.-1.*\x)/1.});
    \begin{scriptsize}
      \draw [fill=uuuuuu] (2.,6.) circle (2.0pt);
      \draw[color=uuuuuu] (2.14,6.33) node {$A$};
      \draw [fill=uuuuuu] (1.,9.) circle (2.0pt);
      \draw[color=uuuuuu] (1.14,9.33) node {$B$};
      \draw [fill=uuuuuu] (8.,0.) circle (2.0pt);
      \draw[color=uuuuuu] (8.14,0.33) node {$C$};
      \draw [fill=uuuuuu] (10.,0.) circle (2.0pt);
      \draw[color=uuuuuu] (10.14,0.33) node {$D$};
    \end{scriptsize}
  \end{tikzpicture}
\end{center}

Wir bringen das Problem erstmal in Normalform:
\begin{equation*}
  \begin{gathered}
    \max \quad z = 2x_1 + x_2\\
    \begin{aligned}
      \textup{unter}
      \quad
      -x_1 - x_2  & \leq -8\\
      -3x_1 - x_2 & \leq -12\\
      x_1 + x_2   & \leq 10\\
      x_1, x_2    & \geq 0
    \end{aligned}
  \end{gathered}
\end{equation*}
Weil einige der $b_i < 0$ sind, können wir nicht einfach $(0,0)$ als erste
Basislösung nehmen. Wir müssen zunächst die erste Basislösung finden. Das ist
die erste Phase der 2-Phasen-Methode. Dafür führen wir die Hilfsvariable $x_0$
wie folgt ein:
\begin{align*}
  \begin{gathered}
    \max \quad z' = -x_0\\
    \begin{aligned}
      \textup{unter}
      \quad
      -x_1 - x_2 - x_0  & \leq -8\\
      -3x_1 - x_2 - x_0 & \leq -12\\
      x_1 + x_2 - x_0   & \leq 10\\
      x_0, x_1, x_2     & \geq 0
    \end{aligned}
  \end{gathered}
\end{align*}
Das ist das Hilfsproblem, das wir lösen müssen, um die erste Basislösung zu
finden. Legen wir los mit den Schlupfvariablen $x_3, x_4, x_5$:
\begin{align*}
  \begin{array}{rcrcrcrcr}
    x_3      & = & -8  & + & x_1   & + & x_2 & + & x_0\\
    x_4      & = & -12 & + & 3 x_1 & + & x_2 & + & x_0\\
    x_5      & = & 10  & - & x_1   & - & x_2 & + & x_0\\
    \midrule
    z'       & = &     &   &       &   &     & - & x_0
  \end{array}
\end{align*}
Die Pivotspalte ist $x_0$. Für die Pivotzeile suchen wir die „unzulässigste“
Zeile. Das ist die Zeile, bei der $b_i$ am kleinsten ist. Dies ist bei $x_4$ der
Fall:
\begin{align*}
  \begin{array}{rcrcrcrcrcr}
    x_3      & = & -8  & + & x_1   & + & x_2 & + & x_0\\
    x_4      & = & -12 & + & 3 x_1 & + & x_2 & + & x_0 & & \leftarrow\\
    x_5      & = & 10  & - & x_1   & - & x_2 & + & x_0\\
    \midrule
    z'       & = &     &   &       &   &     & - & x_0
  \end{array}
\end{align*}
Jetzt tauschen wir $x_4$ gegen $x_0$:
\begin{align*}
  \begin{array}{rcrcrcrcrcr}
    x_3      & = & 4  & - & 2 x_1 &   &     & + & x_4\\
    x_0      & = & 12 & - & 3 x_1 & - & x_2 & + & x_4\\
    x_5      & = & 10 & - & x_1   & - & x_2 & + & x_0\\
    \midrule
    z'       & = &    &   &       &   &     & - & x_0
  \end{array}
\end{align*}
Und ersetzen $x_0$ in den anderen Gleichungen:
\begin{align*}
  \begin{array}{rcrcrcrcrcr}
    x_3      & = & 4   & - & 2 x_1 &   &       & + & x_4\\
    x_0      & = & 12  & - & 3 x_1 & - & x_2   & + & x_4\\
    x_5      & = & 22  & - & 4 x_1 & - & 2 x_2 & + & x_4\\
    \midrule
    z'       & = & -12 & + & 3 x_1 & + & x_2   & - & x_4
  \end{array}
\end{align*}
Ab jetzt wieder Pivotspalte und -zeile bestimmen. Größter Anstieg ist bei $x_1$,
größte Einschränkung in den Zeilen ist bei:
\begin{align*}
  \begin{array}{rcrcrcrcrclr}
    x_3      & = & 4   & - & 2 x_1 &   &       & + & x_4 & \quad & 4/2 = 2            & \leftarrow\\
    x_0      & = & 12  & - & 3 x_1 & - & x_2   & + & x_4 &       & 12/3 = 4\\
    x_5      & = & 22  & - & 4 x_1 & - & 2 x_2 & + & x_4 &       & 22/4 = \num{5,5}\\
    \midrule
    z'       & = & -12 & + & 3 x_1 & + & x_2   & - & x_4
  \end{array}
\end{align*}
Also tauschen wir $x_1$ gegen $x_3$:
\begin{align*}
  \begin{array}{rcrcrcrcrclr}
    x_1      & = & 2  &   &       & - & \frac{1}{2} x_3 & + & \frac{1}{2} x_4\\
    x_0      & = & 6  & - & x_2   & + & \frac{3}{2} x_3 & - & \frac{1}{2} x_4\\
    x_5      & = & 14 & - & 2 x_2 & + & 2 x_3           & - & x_4\\
    \midrule
    z'       & = & -6 & + & x_2   & - & \frac{3}{2} x_3 & + & \frac{1}{2 }x_4
  \end{array}
\end{align*}
Und nochmal: Pivotspalte ist $x_2$, Pivotzeile ist $x_0$. Umtauschen ergibt
\begin{align*}
  \begin{array}{rcrcrcrcrclr}
    x_1      & = & 2 &   &       & - & \frac{1}{2} x_3 & + & \frac{1}{2} x_4\\
    x_2      & = & 6 & - & x_0   & + & \frac{3}{2} x_3 & - & \frac{1}{2} x_4\\
    x_5      & = & 2 & + & 2 x_0 & - & x_3             &   & \\
    \midrule
    z'       & = &   & - & x_0   &   &                 &   &
  \end{array}
\end{align*}
Jetzt haben wir das Hilfs-LP gelöst (Lemma 1.2). Nach diesem hat das Ausgangs-LP
genau dann eine zulässige Lösung, wenn das Hilfs-LP eine optimale Lösung mit
$x_0 = 0$ hat. Wir hätten auch keine weitere Iteration des Simplex-Algorithmus’
mehr machen können, weil sich die Zielfunktion nicht weiter verbessern lässt.
Die erste Basislösung ist jetzt $(x_0, x_1, x_2, x_3, x_4, x_5) = (0, 2, 6, 0,
0, 2)$. Oder graphisch betrachtet bei Punkt $A$:
\begin{center}
  \begin{tikzpicture}[line cap=round,line join=round,>=triangle 45,x=1.0cm,y=1.0cm]
    \draw [color=cqcqcq,, xstep=1.0cm,ystep=1.0cm] (-1.,-1.) grid (11.,10.);
    \draw[->,color=black] (-1.,0.) -- (11.,0.);
    \foreach \x in {1,2,3,4,5,6,7,8,9,10}
    \draw[shift={(\x,0)},color=black] (0pt,2pt) -- (0pt,-2pt) node[below] {\footnotesize $\x$};
    \draw[->,color=black] (0.,-1.) -- (0.,10.);
    \foreach \y in {1,2,3,4,5,6,7,8,9}
    \draw[shift={(0,\y)},color=black] (2pt,0pt) -- (-2pt,0pt) node[left] {\footnotesize $\y$};
    \draw[color=black] (0pt,-10pt) node[right] {\footnotesize $0$};
    \clip(-1.,-1.) rectangle (11.,10.);
    \fill[color=zzttqq,fill=zzttqq,fill opacity=0.10000000149011612] (2.,6.) -- (1.,9.) -- (10.,0.) -- (8.,0.) -- cycle;
    \draw [color=qqqqff,domain=-1.:11.] plot(\x,{(--8.-1.*\x)/1.});
    \draw [color=qqqqff,domain=-1.:11.] plot(\x,{(--12.-3.*\x)/1.});
    \draw [color=qqqqff,domain=-1.:11.] plot(\x,{(--10.-1.*\x)/1.});
    \begin{scriptsize}
      \draw [fill=red] (2.,6.) circle (2.0pt);
      \draw[color=red] (2.14,6.33) node {$A$};
      \draw [fill=uuuuuu] (1.,9.) circle (2.0pt);
      \draw[color=uuuuuu] (1.14,9.33) node {$B$};
      \draw [fill=uuuuuu] (8.,0.) circle (2.0pt);
      \draw[color=uuuuuu] (8.14,0.33) node {$C$};
      \draw [fill=uuuuuu] (10.,0.) circle (2.0pt);
      \draw[color=uuuuuu] (10.14,0.33) node {$D$};
    \end{scriptsize}
  \end{tikzpicture}
\end{center}
Um das Ausgangs-LP zu lösen, müssen wir jetzt nur $x_0$ aus dem Hilfs-LP
streichen (schließlich ist $x_0 = 0$) und können ganz normal mit dem
Simplex-Algorithmus weitermachen. Unsere ursprüngliche Zielfunktion war $z = 2
x_1 + x_2$. Dort müssen wir $x_1$ und $x_2$ ersetzen und erhalten:
\begin{align*}
  \begin{array}{rcrcrcrcrclr}
    x_1      & = & 2  & - & \frac{1}{2} x_3 & + & \frac{1}{2} x_4\\
    x_2      & = & 6  & + & \frac{3}{2} x_3 & - & \frac{1}{2} x_4\\
    x_5      & = & 2  & - & x_3             &   & \\
    \midrule
    z        & = & 10 & + & \frac{1}{2} x_3 & + & \frac{1}{2} x_4
  \end{array}
\end{align*}
Wenn man das weiterrechnet, dann erhält man die optimale Lösung $(x_1, x_2, x_3,
x_4, x_5) = (10, 0, 0, 18, 2)$.

\begin{equation}
  \begin{gathered}
    \max \quad x_1 + x_2\\
    \begin{aligned}
      \textup{unter}
      \quad
      x_1 + 2x_2 + x_3   & = 1\\
      x_1 - 2x_2 - x_4   & = 2\\
      x_1, x_2, x_3, x_4 & \geq 0
    \end{aligned}
  \end{gathered}
\end{equation}

\paragraph{Aufgabe 4}%
\label{par:aufgabe_4}

Ein Landwirt beabsichtigt seinen \SI{10000}{m^2} großen Acker teils mit Blumen
und teils mit Gemüse zu bestellen. Er erwartet pro Quadratmeter Blumen einen
Deckungsbeitrag von \num{2,5} GE zu erzielen, pro Quadratmeter Gemüse \num{1}
GE\@. Mit \num{0,1} GE Pflegekosten pro Quadratmeter Brache muss er rechnen. Die
Beschaffung des Saatguts erfordert Mittel in Höhe von \num{1} GE pro zu
bepflanzendem Quadratmeter Blumen und \num{0,5} GE pro Quadratmeter Gemüse,
wofür insgesamt maximal \num{8000} GE verwendet werden sollen (inklusive
Pflegekosten für evtl. Brachfläche).

Wie viel Acker soll mit Blumen und wie viel mit Gemüse bepflanzt werden, und wie
viel soll brach liegen, um ein optimales Ergebnis zu erreichen?

Ist es besser für den Landwirt, wenn er die Gesamtfläche bestellt?

Sie können eine frei erhältliche Software zur Lösung linearer Programme
benutzen. Beispielsweise \texttt{lpsolve}.

\paragraph{Aufgabe 11}%
\label{par:aufgabe_11}

Leiten Sie das duale LP zu dem LP des klassischen Transportproblems her:

\begin{tabular}{lll}
  $\min$ & $\sum\limits_{i=1}^m \sum\limits_{j=1}^n c_{ij} x_{ij}$\\
  unter & $\sum\limits_{j=1}^n x_{ij} = a_i$ & $(i = 1,2, \ldots, m)$\\
  & $\sum\limits_{i=1}^m x_{ij} = b_j$ & $(j = 1,2, \ldots, n)$\\
  & $x_{ij} \geq 0$ & $(i \leq i \leq m, 1 \leq j \leq n)$
\end{tabular}

\paragraph{Aufgabe 12}%
\label{par:aufgabe_12}
\begin{wrapfigure}{r}{0.2\textwidth}
  $(c_{ij})  =
  \begin{pmatrix}
    5 & 4 & 6\\
    3 & 4 & 2\\
    4 & 3 & 5\\
    4 & 3 & 4
  \end{pmatrix}$
\end{wrapfigure}
Die vier Fertigungsbetriebe $A_1, A_2, A_3, A_4$ eines Unternehmens produzieren
an verschiedenen Orten die gleiche Ware. In jedem Monat werden (in
Mengeneinheiten, ME):
\begin{addmargin}[1cm]{0cm}
  $6$ ME im Betrieb $A_1$, $7$ ME in $A_2$, $8$ ME in $A_3$, $6$ ME in $A_4$
\end{addmargin}
Die Gesamtproduktion soll zu drei verschiedenen Lagerhäusern $B_1, B_2, B_3$
transportiert werden, so dass dort der jeweilige vorhandene Bedarf gedeckt
werden kann:
\begin{addmargin}[1cm]{0cm}
  $B_1$ fragt $10$ ME nach, $B_2$ $8$ ME, $B_3$ $9$ ME.
\end{addmargin}
Die Transportkosten $c_{ij}$ pro Mengeneinheit der Ware von $A_i$ nach $B_j$
sind in der unten stehenden Matrix angegeben.

\begin{enumerate}[(a)]
  \item Bestimmen Sie einen zulässigen Transportplan mit der Nordwesteckenregel.
    Geben Sie die Transportkosten an.

  \item Bestimmen Sie einen zulässigen, möglichst nicht entarteten,
    Transportplan mit der Minimale-Kosten-Regel.
    Geben Sie die Transportkosten an.

    \begin{enumerate}[1.]
      \item $c_{23} = 2$
        \begin{align*}
          \begin{matrix}
                & B_1 & B_2 & B_3 & \\
            A_1 &     &     &     & 6\\
            A_2 &     &     & 7   & 0\\
            A_3 &     &     &     & 8\\
            A_4 &     &     &     & 6\\
                & 10  & 8   & 2
          \end{matrix}
        \end{align*}

      \item $c_{42} = 3$
        \begin{align*}
          \begin{matrix}
                & B_1 & B_2 & B_3 & \\
            A_1 &     &     &     & 6\\
            A_2 &     &     & 7   & 0\\
            A_3 &     &     &     & 8\\
            A_4 &     & 6   &     & 0\\
                & 10  & 2   & 2
          \end{matrix}
        \end{align*}

      \item $c_{32} = 3$
        \begin{align*}
          \begin{matrix}
                & B_1 & B_2 & B_3 & \\
            A_1 &     &     &     & 6\\
            A_2 &     &     & 7   & 0\\
            A_3 &     & 2   &     & 6\\
            A_4 &     & 6   &     & 0\\
                & 10  & 0   & 2
          \end{matrix}
        \end{align*}

      \item $c_{31} = 4$
        \begin{align*}
          \begin{matrix}
                & B_1 & B_2 & B_3 & \\
            A_1 &     &     &     & 6\\
            A_2 &     &     & 7   & 0\\
            A_3 & 6   & 2   &     & 0\\
            A_4 &     & 6   &     & 0\\
                & 4   & 0   & 2
          \end{matrix}
        \end{align*}

      \item $c_{11} = 5$
        \begin{align*}
          \begin{matrix}
                & B_1 & B_2 & B_3 & \\
            A_1 & 4   &     &     & 2\\
            A_2 &     &     & 7   & 0\\
            A_3 & 6   & 2   &     & 0\\
            A_4 &     & 6   &     & 0\\
                & 0   & 0   & 2
          \end{matrix}
        \end{align*}

      \item $c_{13} = 6$
        \begin{align*}
          \begin{matrix}
                & B_1 & B_2 & B_3 & \\
            A_1 & 4   &     & 2   & 0\\
            A_2 &     &     & 7   & 0\\
            A_3 & 6   & 2   &     & 0\\
            A_4 &     & 6   &     & 0\\
                & 0   & 0   & 0
          \end{matrix}
        \end{align*}
        Kosten: 94.
    \end{enumerate}

  \item Bestimmen Sie einen kostenminimalen Transportplan mit der
    $u$-$v$-Methode sowie die zugehörigen Kosten. Verwenden Sie dabei die in (a)
    \underline{oder} (b) gefundene Ausgangslösung.

    Die Basislösung der Nordwesteckenregel lautet
    \begin{align*}
      \begin{matrix}
            & B_1 & B_2 & B_3\\
        A_1 & 6\\
        A_2 & 4   & 3\\
        A_3 &     & 5   & 3\\
        A_4 &     &     & 6
      \end{matrix}
    \end{align*}
    Das hat die folgenden Basisvariablen:
    \begin{align*}
      B & = \left\{ (1,1), (2,1), (2,2), (3,2), (3,3), (4,3) \right\}
    \end{align*}
    Diese führen zu folgenden Gleichungen $(u_i + v_j = c_{ij})$:
    \begin{align*}
      u_1 + v_1 & = 5 & u_3 + v_2 & = 3\\
      u_2 + v_1 & = 3 & u_3 + v_3 & = 5\\
      u_2 + v_2 & = 4 & u_4 + v_3 & = 4
    \end{align*}
    Wir können eine Variable frei wählen, also setzen wir $u_2 = 0$, woraus
    folgt: $v_1 = 3, v_2 = 4, u_1 = 2, u_3 = -1, v_3 = 6, u_4 = -2$.

    Jetzt prüfen wir die duale Zulässigkeit $\overline{c_{ij}} = c_{ij} - u_i -
    v_j \geq 0$:
    \begin{align*}
      \overline{c_{12}} & = 4 - 2 - 4 = -2 \quad \falsch\\
      \overline{c_{13}} & = 6 - 2 - 6 = -2 \quad \falsch\\
      \overline{c_{23}} & = 2 - 0 - 6 = -4 \quad \falsch\\
      \overline{c_{31}} & = 4 - (-1) - 3 = 2 \quad \richtig\\
      \overline{c_{41}} & = 4 - (-2) - 3 = 3 \quad \richtig\\
      \overline{c_{42}} & = 3 - (-2) - 4 = 1 \quad \richtig
    \end{align*}
    Das Minimum dieser $\overline{c_{ij}}$ ist $\overline{c_{23}}$, also wird
    $(2,3)$ in die Basis aufgenommen. Aber welche Position verlässt die Basis?
    Schauen wir uns den Graph an:
    \begin{center}
      \begin{tikzpicture}
        \node (a4) at (0,0) [shape=circle, draw] {$A_4$};
        \node (a3) at (0,2) [shape=circle, draw] {$A_3$};
        \node (a2) at (0,4) [shape=circle, draw] {$A_2$};
        \node (a1) at (0,6) [shape=circle, draw] {$A_1$};
        \node (b3) at (5,1) [shape=circle, draw] {$B_3$};
        \node (b2) at (5,3) [shape=circle, draw] {$B_2$};
        \node (b1) at (5,5) [shape=circle, draw] {$B_1$};

        \draw (a1) -- (b1) -- (a2) -- (b2) -- (a3) -- (b3) -- (a4);
        \draw [red] (a2) -- (b3);
      \end{tikzpicture}
    \end{center}
    Die rote Linie von $A_2$ nach $B_3$ zeigt die neue Basisvariable. Mit dieser
    neuen Kante finden wir den Kreis: $A_2 - B_2 - A_3 - B_3 - A_2$.
    \begin{align*}
      \begin{matrix}
        & B_1 & B_2  & B_3\\
        A_1 & 6\\
        A_2 & 4
        & ^- 3 \ \tikz[remember picture] \draw [blue!50, fill=blue!50] circle (1pt) node (n1) {};
        & \tikz[remember picture] \draw [blue!50, fill=blue!50] circle (1pt) node (n2) {}; \ \phantom{3}^+\\
        A_3 &
        & ^+ 5 \ \tikz[remember picture] \draw [blue!50, fill=blue!50] circle (1pt) node (n4) {};
        & \tikz[remember picture] \draw [blue!50, fill=blue!50] circle (1pt) node (n3) {}; \ 3^-\\
        A_4 &     &      & 6
      \end{matrix}
      &
      \begin{tikzpicture}[remember picture, overlay]
        \draw [blue!50] (n1.center)
          -- (n2.center)
          -- (n3.center)
          -- (n4.center)
          -- (n1.center);
      \end{tikzpicture}
    \end{align*}
    Von den mit $(-)$ markierten Einträgen nehmen wir jetzt das Minimum (in
    diesem Fall: $3$) und ziehen es von den $(-)$ markierten Einträgen ab und
    addieren es auf die mit $(+)$ markierten Einträge auf.

    Das führt zu folgendem Transportplan:
    \begin{align*}
      \begin{matrix}
            & B_1 & B_2 & B_3\\
        A_1 & 6\\
        A_2 & 4   & 0   & 3\\
        A_3 &     & 8   & 0\\
        A_4 &     &     & 6
      \end{matrix}
    \end{align*}
    Von den beiden Nullen darf aber nur eine bleiben, da eine der Basisvariablen
    entfernt werden und eine andere behalten werden muss. Welche bleibt und
    welche entfernt wird, ist egal. Wir wählen $x_{22}$ als bleibende, also wird
    $x_{33}$ entfernt und wir erhalten folgenden Transportplan:
    \begin{align*}
      \begin{matrix}
            & B_1 & B_2 & B_3\\
        A_1 & 6\\
        A_2 & 4   & 0   & 3\\
        A_3 &     & 8\\
        A_4 &     &     & 6
      \end{matrix}
    \end{align*}
    Hier wird wieder optimiert. Wir nehmen die Basisvariablen her:
    \begin{align*}
      B & = \left\{ (1,1), (2,1), (2,2), (2,3), (3,2), (4,3) \right\}
    \end{align*}
    Diese führen zu den $u,v$-Gleichungen:
    \begin{align*}
      u_1 + v_1 & = 5 & u_2 + v_3 & = 2\\
      u_2 + v_1 & = 3 & u_3 + v_2 & = 3\\
      u_2 + v_2 & = 4 & u_4 + v_3 & = 4
    \end{align*}
    Wir wählen $u_2 = 0$ und erhalten: $v_1 = 3, v_2 = 4, v_3 = 2, u_1 = 2, u_3
    = -1, u_4 = 2$. Dann prüfen wir wieder die duale Zulässigkeit:
    \begin{align*}
      \overline{c_{12}} & = 4 - 2 - 4    = -2 \quad \falsch\\
      \overline{c_{13}} & = 6 - 2 - 2    = 2 \quad \richtig\\
      \overline{c_{31}} & = 4 - (-1) - 3 = 2 \quad \richtig\\
      \overline{c_{33}} & = 5 - (-1) - 2 = 4 \quad \richtig\\
      \overline{c_{41}} & = 4 - 2 - 3    = -1 \quad \falsch\\
      \overline{c_{42}} & = 3 - 2 - 4    = -3 \quad \falsch
    \end{align*}
    Das Minimum ist hier $\overline{c_{42}} = -3$, also wird $(4,2)$ in die
    Basis aufgenommen. Wir suchen wieder einen Kreis im Transportplan:
    \begin{center}
      \begin{tikzpicture}
        \node (a4) at (0,0) [shape=circle, draw] {$A_4$};
        \node (a3) at (0,2) [shape=circle, draw] {$A_3$};
        \node (a2) at (0,4) [shape=circle, draw] {$A_2$};
        \node (a1) at (0,6) [shape=circle, draw] {$A_1$};
        \node (b3) at (5,1) [shape=circle, draw] {$B_3$};
        \node (b2) at (5,3) [shape=circle, draw] {$B_2$};
        \node (b1) at (5,5) [shape=circle, draw] {$B_1$};

        \draw (a1) -- (b1) -- (a2);
        \draw (b2) -- (a2) -- (b3);
        \draw (a3) -- (b2);
        \draw (a4) -- (b3);
        \draw [red] (a4) -- (b2);
      \end{tikzpicture}
    \end{center}
    Wir finden den Kreis $A_2 - B_2 - A_4 - B_3 - A_2$ und stellen diesen im
    Transportplan dar:
    \begin{align*}
      \begin{matrix}
            & B_1 & B_2 & B_3\\
        A_1 & 6\\
        A_2 & 4
            & {}^- 0 \ \tikz[remember picture] \draw [blue!50, fill=blue!50] circle (1pt) node (n1) {};
            & \tikz[remember picture] \draw [blue!50, fill=blue!50] circle (1pt) node (n2) {}; \ 3^+\\
        A_3 &     & 8   & \\
        A_4 &
            & {}^+\phantom{0}\ \tikz[remember picture] \draw [blue!50, fill=blue!50] circle (1pt) node (n4) {};
            & \tikz[remember picture] \draw [blue!50, fill=blue!50] circle (1pt) node (n3) {}; \ 6^-
      \end{matrix}
      &
      \begin{tikzpicture}[remember picture, overlay]
        \draw [blue!50] (n1.center)
          -- (n2.center)
          -- (n3.center)
          -- (n4.center)
          -- (n1.center);
      \end{tikzpicture}
    \end{align*}
    Das Minimum der mit $(-)$ markierten Zellen ist $0$. Was ändert sich also?
    Eigentlich nichts, nur dass die Basisvariablen wechseln: $(2,2)$ geht raus,
    $(4,2)$ geht rein. Wir erhalten folgenden Transportplan:
    \begin{align*}
      \begin{matrix}
             & B_1 & B_2 & B_3\\
        A_1  & 6\\
        A_2  & 4   &     & 3\\
        A_3  &     & 8   & \\
        A_4  &     & 0   & 6
      \end{matrix}
    \end{align*}
    Jetzt wieder die Gleichungen aufstellen und die duale Zulässigkeit prüfen
    (spare ich mir hier). Das Minimum der Unzulässigkeiten findet sich bei
    $\overline{c_{41}}$, was zu folgendem Graphen führt:
    \begin{center}
      \begin{tikzpicture}
        \node (a4) at (0,0) [shape=circle, draw] {$A_4$};
        \node (a3) at (0,2) [shape=circle, draw] {$A_3$};
        \node (a2) at (0,4) [shape=circle, draw] {$A_2$};
        \node (a1) at (0,6) [shape=circle, draw] {$A_1$};
        \node (b3) at (5,1) [shape=circle, draw] {$B_3$};
        \node (b2) at (5,3) [shape=circle, draw] {$B_2$};
        \node (b1) at (5,5) [shape=circle, draw] {$B_1$};

        \draw (a1) -- (b1) -- (a2);
        \draw (a2) -- (b3);
        \draw (a3) -- (b2);
        \draw (a4) -- (b3);
        \draw (a4) -- (b2);
        \draw [red] (a4) -- (b1);
      \end{tikzpicture}
    \end{center}
    Im Transportplan finden wir folgenden Kreis:
    \begin{align*}
      \begin{matrix}
            & B_1 & B_2 & B_3\\
        A_1 & 6\\
        A_2 & {}^-4 \ \tikz[remember picture] \draw [blue!50, fill=blue!50] circle (1pt) node (n1) {};
            &
            & \tikz[remember picture] \draw [blue!50, fill=blue!50] circle (1pt) node (n2) {}; \ 3^+\\
        A_3 &     & 8   & \\
        A_4 & {}^+\phantom{0}\ \tikz[remember picture] \draw [blue!50, fill=blue!50] circle (1pt) node (n4) {};
            & 0
            & \tikz[remember picture] \draw [blue!50, fill=blue!50] circle (1pt) node (n3) {}; \ 6^-
      \end{matrix}
      &
      \begin{tikzpicture}[remember picture, overlay]
        \draw [blue!50] (n1.center)
          -- (n2.center)
          -- (n3.center)
          -- (n4.center)
          -- (n1.center);
      \end{tikzpicture}
    \end{align*}
    Jetzt ändert sich auch was am Transportplan! $\delta = 4$ wird von den mit
    $(-)$ markierten Zellen subtrahiert und auf die mit $(+)$ markierten Zellen
    addiert. Das ergibt:
    \begin{align*}
      \begin{matrix}
             & B_1 & B_2 & B_3\\
        A_1  & 6\\
        A_2  &     &     & 7\\
        A_3  &     & 8   & \\
        A_4  & 4   & 0   & 2
      \end{matrix}
    \end{align*}
    Stellt man wieder die $u,v$-Gleichungen auf und überprüft die duale
    Zulässigkeit, dann stellt man fest, dass dieser Plan mit den Kosten $92$
    optimal ist:
    \begin{align*}
      B & = \left\{ (1,1), (2,3), (3,2), (4,1), (4,2), (4,3) \right\}
    \end{align*}
    Diese führen zu den $u,v$-Gleichungen:
    \begin{align*}
      u_1 + v_1 & = 5 & u_4 + v_1 & = 4\\
      u_2 + v_3 & = 2 & u_4 + v_2 & = 3\\
      u_3 + v_2 & = 3 & u_4 + v_3 & = 4
    \end{align*}
    Wir wählen $u_1 = 0$ und erhalten:
    $u_2 = -3, u_3 = -1, u_4 = -1, v_1 = 5, v_2 = 4, v_3 = 5$.
    \begin{align*}
      \overline{c_{12}} & = 4 - 0 - 4 = 0 \quad \richtig\\
      \overline{c_{13}} & = 6 - 0 - 5 = 1 \quad \richtig\\
      \overline{c_{21}} & = 3 - (-3) - 5 = 1 \quad \richtig\\
      \overline{c_{22}} & = 4 - (-3) - 4 = 3 \quad \richtig\\
      \overline{c_{31}} & = 4 - (-1) - 5 = 0 \quad \richtig\\
      \overline{c_{33}} & = 5 - (-1) - 5 = 1 \quad \richtig
    \end{align*}

    Wenn man die Basislösung der MK-Methode nimmt, dann kommt man auf folgende
    Menge von Basisvariablen:
    \begin{align*}
      B = \{(1,1), (1,3), (2,3), (3,1), (3,2), (4,2)\}
    \end{align*}
    Aus diesen ergeben sich die folgenden $u,v$-Gleichungen:
    \begin{align*}
      u_1 + v_1 & = 5 & u_3 + v_1 & = 4\\
      u_1 + v_3 & = 6 & u_3 + v_2 & = 3\\
      u_2 + v_3 & = 2 & u_4 + v_2 & = 3
    \end{align*}
    Aus $u_1 = 0$ folgt:
    $v_1 = 5, v_3 = 6, u_2 = -4, u_3 = -1, v_2 = 4, u_4 = -1$.
    Duale Zulässigkeit prüfen:
    \begin{align*}
      \overline{c_{12}} & = 4 - 0 - 4 = 0 \quad \richtig\\
      \overline{c_{21}} & = 3 - (-4) - 5 = 2 \quad \richtig\\
      \overline{c_{22}} & = 4 - (-4) - 4 = 4 \quad \richtig\\
      \overline{c_{33}} & = 5 - (-1) - 6 = 0 \quad \richtig\\
      \overline{c_{41}} & = 4 - (-1) - 5 = 0 \quad \richtig\\
      \overline{c_{43}} & = 4 - (-1) - 6 = -1 \quad \falsch
    \end{align*}

    \begin{minipage}[t]{0.2\textwidth}
      vorher:
      \begin{align*}
        \begin{matrix}
               & B_1 & B_2 & B_3\\
          A_1
          & 4^+ \tikz[remember picture] \draw [blue!50, fill=blue!50] circle (1pt) node (n1) {};
          &
          & \tikz[remember picture] \draw [blue!50, fill=blue!50] circle (1pt) node (n2) {};\ 2^-\\
          A_2  &     &     & \ 7\\
          A_3
          & 6^- \tikz[remember picture] \draw [blue!50, fill=blue!50] circle (1pt) node (n3) {};
          & 2^+ \tikz[remember picture] \draw [blue!50, fill=blue!50] circle (1pt) node (n4) {};
          & \\
          A_4
          &
          & 6^- \tikz[remember picture] \draw [blue!50, fill=blue!50] circle (1pt) node (n5) {};
          & \tikz[remember picture] \draw [blue!50, fill=blue!50] circle (1pt) node (n6) {}; \ \phantom{0}^+
        \end{matrix}
        &
        \begin{tikzpicture}[remember picture, overlay]
          \draw [blue!50] (n1.center)
          -- (n2.center)
          -- (n6.center)
          -- (n5.center)
          -- (n4.center)
          -- (n3.center)
          -- (n1.center);
        \end{tikzpicture}
      \end{align*}
      $\delta = 2$
    \end{minipage}
    \begin{minipage}[t]{0.4\textwidth}
      Graph des Transportplans:
      \begin{center}
        \begin{tikzpicture}
          \node (a4) at (0,0) [shape=circle, draw] {$A_4$};
          \node (a3) at (0,2) [shape=circle, draw] {$A_3$};
          \node (a2) at (0,4) [shape=circle, draw] {$A_2$};
          \node (a1) at (0,6) [shape=circle, draw] {$A_1$};
          \node (b3) at (5,1) [shape=circle, draw] {$B_3$};
          \node (b2) at (5,3) [shape=circle, draw] {$B_2$};
          \node (b1) at (5,5) [shape=circle, draw] {$B_1$};

          \draw (a1) -- (b1);
          \draw (a1) -- (b3);
          \draw (a2) -- (b3);
          \draw (a3) -- (b1);
          \draw (a3) -- (b2);
          \draw (a4) -- (b2);
          \draw [red] (a4) -- (b3);
        \end{tikzpicture}
      \end{center}
    \end{minipage}
    \begin{minipage}[t]{0.2\textwidth}
      nachher:
      \begin{align*}
        \begin{matrix}
              & B_1 & B_2 & B_3\\
          A_1 & 6   &     & \\
          A_2 &     &     & 7\\
          A_3 & 4   & 4   & \\
          A_4 &     & 4   & 2
        \end{matrix}
      \end{align*}
      Kosten: 92. Nach weiterer Prüfung der dualen Zulässigkeit ergibt sich die
      Optimalität.
    \end{minipage}
\end{enumerate}

\paragraph{Aufgabe 13}%
\label{par:aufgabe_13}

Ein Unternehmen hat für die Besetzung von vier Stellen $A_1, \ldots, A_4$ fünf
Bewerber $B_1, \ldots, B_5$ zur Verfügung, die für einzelne Stellen
unterschiedlich geeignet sind. Die Eignungspunkte (von $1 = $ nicht sehr
geeignet bis $10 =$ am besten geeignet) der Bewerber wurden wie folgt
festgestellt:
\begin{center}
  \begin{tabular}{c|ccccc}
           & $B_1$ & $B_2$ & $B_3$ & $B_4$ & $B_5$\\
    \midrule
    $A_1$  & $5$   & $6$   & $8$   & $6$   & $4$\\
    $A_2$  & $7$   & $5$   & $10$  & $7$   & $3$\\
    $A_3$  & $4$   & $6$   & $5$   & $3$   & $6$\\
    $A_4$  & $4$   & $4$   & $5$   & $3$   & $5$\\
  \end{tabular}
\end{center}

Welche Zuordnung hat \underline{maximal} gesamte Eignungspunkte (und sollte
vorgenommen werden)? Führen Sie notwendige Maßnahmen durch und lösen Sie das
Problem mit der Ungarischen Methode.

Führen wir zuerst die notwendigen Maßnahmen durch. Bevor wir die Ungarische
Methode anwenden können, müssen wir das Problem in die richtige Form bringen.
Die richtige Form hat folgende Eigenschaften: es gibt $n$ Arbeiter $A_1, \ldots
A_n$, $n$ Tätigkeiten $B_1, \ldots, B_n$ und es gibt Ausführungskosten $c_{ij}$,
welche minimiert werden sollen.

In unserem Beispiel haben wir aber einerseits $4$ Tätigkeiten (Stellen) und $5$
Bewerber und andererseits sind Eignungspunkte gegeben, die maximiert werden
sollen.

Um das Problem der unterschiedlichen Anzahl von Tätigkeiten und Bewerbern zu
lösen, führen wir eine neue Tätigkeit $A_5$ ein und setzen die Eignungspunkte
für alle Bewerber auf $0$. Damit erhalten wir folgende Tabelle:

\begin{center}
  \begin{tabular}{c|ccccc}
           & $B_1$ & $B_2$ & $B_3$ & $B_4$ & $B_5$\\
    \midrule
    $A_1$  & $5$   & $6$   & $8$   & $6$   & $4$\\
    $A_2$  & $7$   & $5$   & $10$  & $7$   & $3$\\
    $A_3$  & $4$   & $6$   & $5$   & $3$   & $6$\\
    $A_4$  & $4$   & $4$   & $5$   & $3$   & $5$\\
    $A_5$  & $0$   & $0$   & $0$   & $0$   & $0$\\
  \end{tabular}
\end{center}

Für die Problematik der Maximierung der Eignungspunkte/Minimierung der
Ausführungskosten multiplizieren wir alle Einträge mit $-1$ und erhalten diese
Tabelle:

\begin{center}
  \begin{tabular}{c|ccccc}
            & $B_1$ & $B_2$ & $B_3$ & $B_4$ & $B_5$\\
    \midrule
    $A_1$   & $-5$  & $-6$  & $-8$  & $-6$  & $-4$\\
    $A_2$   & $-7$  & $-5$  & $-10$ & $-7$  & $-3$\\
    $A_3$   & $-4$  & $-6$  & $-5$  & $-3$  & $-6$\\
    $A_4$   & $-4$  & $-4$  & $-5$  & $-3$  & $-5$\\
    $A_5$   & $0$   & $0$   & $0$   & $0$   & $0$\\
  \end{tabular}
\end{center}

Jetzt können wir mit der Ungarischen Methode loslegen. Wir beginnen mit Schritt
\begin{enumerate}[(1)]
  \item Initialisierung: Bilde aus $(c_{ij})$ die reduzierte Kostenmatrix $R$
    wie folgt
    \begin{itemize}
      \item aus jeder Zeile wird von allen Elementen der Zeile das kleinste
        Element subtrahiert;
      \item aus jeder Spalte wird von allen Elementen der Spalte das kleinste
        Element subtrahiert.
    \end{itemize}
    ($R$ hat in also in jeder Zeile und in jeder Spalte mindestens ein Null).
\end{enumerate}

Wir bestimmen wir die Zeilenminima und subtrahieren sie von jedem Zeilenelement:

\begin{center}
  \begin{tabular}{c|ccccc|c}
             & $B_1$ & $B_2$ & $B_3$ & $B_4$ & $B_5$ & $\min_{\text{Zeile}}$\\
    \midrule
    $A_1$    & $-5$  & $-6$  & $-8$  & $-6$  & $-4$  & $-8$\\
    $A_2$    & $-7$  & $-5$  & $-10$ & $-7$  & $-3$  & $-10$\\
    $A_3$    & $-4$  & $-6$  & $-5$  & $-3$  & $-6$  & $-6$\\
    $A_4$    & $-4$  & $-4$  & $-5$  & $-3$  & $-5$  & $-5$\\
    $A_5$    & $0$   & $0$   & $0$   & $0$   & $0$   & $0$\\
  \end{tabular}
  \qquad $\leadsto$ \qquad
  \begin{tabular}{c|ccccc}
              & $B_1$ & $B_2$ & $B_3$ & $B_4$ & $B_5$\\
    \midrule
    $A_1$     & $3$   & $2$   & $0$   & $2$   & $4$\\
    $A_2$     & $3$   & $5$   & $0$   & $3$   & $7$\\
    $A_3$     & $2$   & $0$   & $1$   & $3$   & $0$\\
    $A_4$     & $1$   & $1$   & $0$   & $2$   & $0$\\
    $A_5$     & $0$   & $0$   & $0$   & $0$   & $0$\\
  \end{tabular}
\end{center}

Jetzt müssten wir die Spaltenminima bestimmen und von jedem Spaltenelement
abziehen, aber da die Spaltenminima durch die letzte Zeile immer $0$ sind,
ändert das nichts. Das bringt uns zu Schritt
\begin{enumerate}[(2)]
  \item Bestimme die minimale Zahl $d(R)$ der Zeilen und Spalten, die alle
    Nullen von $R$ überdecken (sog. Decklinien).
\end{enumerate}
Das bringt uns zu einem neuen Problem: die Anzahl der Decklinien finden. Fangen
wir an mit Schritt
\begin{enumerate}[(D1)]
  \item Wähle unabhängige Nullen aus (z.\,B. spaltenweise); diese sind
    Zuordnungen.
\end{enumerate}
Wir könnten beispielsweise die folgenden Nullen auswählen:
\begin{center}
  \begin{tabular}{c|ccccc}
                & $B_1$      & $B_2$      & $B_3$      & $B_4$ & $B_5$\\
    \midrule
    $A_1$       & $3$        & $2$        & \fbox{$0$} & $2$   & $4$\\
    $A_2$       & $3$        & $5$        & $0$        & $3$   & $7$\\
    $A_3$       & $2$        & \fbox{$0$} & $1$        & $3$   & $0$\\
    $A_4$       & $1$        & $1$        & $0$        & $2$   & \fbox{$0$}\\
    $A_5$       & \fbox{$0$} & $0$        & $0$        & $0$   & $0$\\
  \end{tabular}
\end{center}
Weiter geht’s mit Schritt
\begin{enumerate}[(D2)]
  \item Markiere Zeilen ohne Zuordnungen (Zeilen ohne ausgewählte Nullen).
\end{enumerate}
Das wäre die Zeile $A_2$:
\begin{center}
  \begin{tabular}{c|cccccc}
                   & $B_1$      & $B_2$      & $B_3$      & $B_4$ & $B_5$\\
    \cmidrule{1-6}
    $A_1$          & $3$        & $2$        & \fbox{$0$} & $2$   & $4$\\
    $A_2$          & $3$        & $5$        & $0$        & $3$   & $7$          & $\leftarrow$\\
    $A_3$          & $2$        & \fbox{$0$} & $1$        & $3$   & $0$\\
    $A_4$          & $1$        & $1$        & $0$        & $2$   & \fbox{$0$}\\
    $A_5$          & \fbox{$0$} & $0$        & $0$        & $0$   & $0$\\
  \end{tabular}
\end{center}
Jetzt wird’s langsam schwierig:
\begin{enumerate}[(D3)]
  \item Markiere die noch nicht markierten Spalten, welche Nullen in den bereits
    markierten Zeilen haben.
\end{enumerate}
Da $A_2$ die einzige markierte Zeile ist und diese nur eine Null in Spalte $B_3$
hat, markieren wir $B_3$:
\begin{center}
  \begin{tabular}{c|cccccc}
    \multicolumn{3}{c}{} & $\downarrow$\\
                         & $B_1$          & $B_2$      & $B_3$      & $B_4$ & $B_5$\\
    \cmidrule{1-6}
    $A_1$                & $3$            & $2$        & \fbox{$0$} & $2$   & $4$\\
    $A_2$                & $3$            & $5$        & $0$        & $3$   & $7$          & $\leftarrow$\\
    $A_3$                & $2$            & \fbox{$0$} & $1$        & $3$   & $0$\\
    $A_4$                & $1$            & $1$        & $0$        & $2$   & \fbox{$0$}\\
    $A_5$                & \fbox{$0$}     & $0$        & $0$        & $0$   & $0$\\
  \end{tabular}
\end{center}
Nach der Spaltenmarkierung geht’s weiter mit der Zeilenmarkierung:
\begin{enumerate}[(D4)]
  \item Markiere die noch nicht markierten Zeilen, welche Zuordnungen in den
    markierten Spalten haben.
\end{enumerate}
Unsere einzige markierte Spalte ist $B_3$. Dort gibt es eine Zuordnung in der
ersten Zeile, also markieren wir die Zeile $A_1$:
\begin{center}
  \begin{tabular}{c|cccccc}
    \multicolumn{3}{c}{}  & $\downarrow$\\
                          & $B_1$          & $B_2$      & $B_3$      & $B_4$ & $B_5$\\
    \cmidrule{1-6}
    $A_1$                 & $3$            & $2$        & \fbox{$0$} & $2$   & $4$          & $\leftarrow$\\
    $A_2$                 & $3$            & $5$        & $0$        & $3$   & $7$          & $\leftarrow$\\
    $A_3$                 & $2$            & \fbox{$0$} & $1$        & $3$   & $0$\\
    $A_4$                 & $1$            & $1$        & $0$        & $2$   & \fbox{$0$}\\
    $A_5$                 & \fbox{$0$}     & $0$        & $0$        & $0$   & $0$\\
  \end{tabular}
\end{center}
Die letzten zwei Schritte, also (D3) und (D4), wiederholen wir solange, bis
keine Markierungen mehr auftreten. Glücklicherweise gibt es keine weiteren
Nullen in den bereits markierten Zeilen, sodass wir keine weiteren Spalten
markieren müssen.

Die Anzahl der Decklinien $d(R)$ ergibt sich jetzt aus der Summe der Anzahl der
nichtmarkierten Zeilen und der markierten Spalten. In unserem Fall haben wir $3$
nichtmarkierte Zeilen ($A_3, A_4$ und $A_5$) und eine markierte Spalte ($B_3$),
sodass $d(R) = 3+1 = 4$ ist. Zeichnen wir diese Decklinien einmal ein:
\begin{center}
  \begin{tabular}{cc|cccccc}
    &
    & \tikz[remember picture] \node (B1_top) {$B_1$};
    & \tikz[remember picture] \node (B2_top) {$B_2$};
    & \tikz[remember picture] \node (B3_top) {$B_3$};
    & \tikz[remember picture] \node (B4_top) {$B_4$};
    & \tikz[remember picture] \node (B5_top) {$B_5$};
    \\

    \cmidrule{2-7}
    \tikz[remember picture] \node (A1_left) {};
    & $A_1$
    & $3$
    & $2$
    & \fbox{$0$}
    & $2$
    & $4$
    & \tikz[remember picture] \node (A1_right) {};
    \\

    \tikz[remember picture] \node (A2_left) {};
    & $A_2$
    & $3$
    & $5$
    & $0$
    & $3$
    & $7$
    & \tikz[remember picture] \node (A2_right) {};
    \\

    \tikz[remember picture] \node (A3_left) {};
    & $A_3$
    & $2$
    & \fbox{$0$}
    & $1$
    & $3$
    & $0$
    & \tikz[remember picture] \node (A3_right) {};
    \\

    \tikz[remember picture] \node (A4_left) {};
    & $A_4$
    & $1$
    & $1$
    & $0$
    & $2$
    & \fbox{$0$}
    & \tikz[remember picture] \node (A4_right) {};
    \\

    \tikz[remember picture] \node (A5_left) {};
    & $A_5$
    & \fbox{$0$}
    & $0$
    & $0$
    & $0$
    & $0$
    & \tikz[remember picture] \node (A5_right) {};
    \\

    \multicolumn{2}{c}{}
    & \tikz[remember picture] \node (B1_bottom) {};
    & \tikz[remember picture] \node (B2_bottom) {};
    & \tikz[remember picture] \node (B3_bottom) {};
    & \tikz[remember picture] \node (B4_bottom) {};
    & \tikz[remember picture] \node (B5_bottom) {};
  \end{tabular}

  \begin{tikzpicture}[remember picture, overlay]
    \draw [blue!50] (B3_top.south) -- (B3_bottom.center);

    \draw [blue!50] (A3_left.center) -- (A3_right.center);
    \draw [blue!50] (A4_left.center) -- (A4_right.center);
    \draw [blue!50] (A5_left.center) -- (A5_right.center);
  \end{tikzpicture}
\end{center}
Von den nicht überdeckten Elementen ist $2$ das Minimum.

Kommen wir zum nächsten Schritt der Ungarischen Methode:
\begin{enumerate}[(3)]
  \item Falls $d(R) = n$: fertig, optimale Lösung ist ablesbar: Für $n$
    „unabhängige“ $r_{ij} = 0$ setze $x_{ij} = 1$, die übrigen $x_{ij} = 0$.
\end{enumerate}
Leider ist unser $d(R) = 4 < 5$ und das heißt, dass wir noch nicht fertig sind.
Daher gehen wir weiter zum nächsten Schritt:
\begin{enumerate}[(4)]
  \item Falls $d(R) < n$: Aktualisiere $R$ wie folgt
    \begin{itemize}
      \item bestimme $r^* = \min \left\{ r_{ij} \mid r_{ij} \text{ ist nicht
        überdeckt} \right\}$;
      \item subtrahiere $r^*$ von allen nicht überdeckten Elementen;
      \item addiere $r^*$ zu allen doppelt überdeckten Elementen;
      \item die nur einmal überdeckten Elemente bleiben unverändert.
    \end{itemize}
\end{enumerate}
Wir erhalten $r^* = 2$. Dieses subtrahieren wir von allen nicht überdeckten
Elementen. Wir erhalten:

\begin{center}
  \begin{tabular}{cc|cccccc}
    &
    & \tikz[remember picture] \node (B1_top) {$B_1$};
    & \tikz[remember picture] \node (B2_top) {$B_2$};
    & \tikz[remember picture] \node (B3_top) {$B_3$};
    & \tikz[remember picture] \node (B4_top) {$B_4$};
    & \tikz[remember picture] \node (B5_top) {$B_5$};
    \\

    \cmidrule{2-7}
    \tikz[remember picture] \node (A1_left) {};
    & $A_1$
    & $3$
    & $2$
    & \fbox{$0$}
    & $2$
    & $4$
    & \tikz[remember picture] \node (A1_right) {};
    \\

    \tikz[remember picture] \node (A2_left) {};
    & $A_2$
    & $3$
    & $5$
    & $0$
    & $3$
    & $7$
    & \tikz[remember picture] \node (A2_right) {};
    \\

    \tikz[remember picture] \node (A3_left) {};
    & $A_3$
    & $2$
    & \fbox{$0$}
    & $1$
    & $3$
    & $0$
    & \tikz[remember picture] \node (A3_right) {};
    \\

    \tikz[remember picture] \node (A4_left) {};
    & $A_4$
    & $1$
    & $1$
    & $0$
    & $2$
    & \fbox{$0$}
    & \tikz[remember picture] \node (A4_right) {};
    \\

    \tikz[remember picture] \node (A5_left) {};
    & $A_5$
    & \fbox{$0$}
    & $0$
    & $0$
    & $0$
    & $0$
    & \tikz[remember picture] \node (A5_right) {};
    \\

    \multicolumn{2}{c}{}
    & \tikz[remember picture] \node (B1_bottom) {};
    & \tikz[remember picture] \node (B2_bottom) {};
    & \tikz[remember picture] \node (B3_bottom) {};
    & \tikz[remember picture] \node (B4_bottom) {};
    & \tikz[remember picture] \node (B5_bottom) {};
  \end{tabular}
  \begin{tikzpicture}[remember picture, overlay]
    \draw [blue!50] (B3_top.south) -- (B3_bottom.center);

    \draw [blue!50] (A3_left.center) -- (A3_right.center);
    \draw [blue!50] (A4_left.center) -- (A4_right.center);
    \draw [blue!50] (A5_left.center) -- (A5_right.center);
  \end{tikzpicture}
  $(-r^*) \leadsto$
  \begin{tabular}{cc|cccccc}
    &
    & \tikz[remember picture] \node (B1_top) {$B_1$};
    & \tikz[remember picture] \node (B2_top) {$B_2$};
    & \tikz[remember picture] \node (B3_top) {$B_3$};
    & \tikz[remember picture] \node (B4_top) {$B_4$};
    & \tikz[remember picture] \node (B5_top) {$B_5$};
    \\

    \cmidrule{2-7}
    \tikz[remember picture] \node (A1_left) {};
    & $A_1$
    & \underline{$1$}
    & \underline{$0$}
    & $0$
    & \underline{$0$}
    & \underline{$2$}
    & \tikz[remember picture] \node (A1_right) {};
    \\

    \tikz[remember picture] \node (A2_left) {};
    & $A_2$
    & \underline{$1$}
    & \underline{$3$}
    & $0$
    & \underline{$1$}
    & \underline{$5$}
    & \tikz[remember picture] \node (A2_right) {};
    \\

    \tikz[remember picture] \node (A3_left) {};
    & $A_3$
    & $2$
    & $0$
    & $1$
    & $3$
    & $0$
    & \tikz[remember picture] \node (A3_right) {};
    \\

    \tikz[remember picture] \node (A4_left) {};
    & $A_4$
    & $1$
    & $1$
    & $0$
    & $2$
    & $0$
    & \tikz[remember picture] \node (A4_right) {};
    \\

    \tikz[remember picture] \node (A5_left) {};
    & $A_5$
    & $0$
    & $0$
    & $0$
    & $0$
    & $0$
    & \tikz[remember picture] \node (A5_right) {};
    \\

    \multicolumn{2}{c}{}
    & \tikz[remember picture] \node (B1_bottom) {};
    & \tikz[remember picture] \node (B2_bottom) {};
    & \tikz[remember picture] \node (B3_bottom) {};
    & \tikz[remember picture] \node (B4_bottom) {};
    & \tikz[remember picture] \node (B5_bottom) {};
  \end{tabular}
\end{center}

Zuletzt müssen wir $r^*$ noch auf alle doppelt überdeckten Elemente addieren.
Zur Hilfe gibt’s links den aktuellen Stand mit Decklinien und rechts das
Ergebnis ohne Decklinien aber mit unterstrichenen Zahlen, die sich verändert
haben:
\begin{center}
  \begin{tabular}{cc|cccccc}
    &
    & \tikz[remember picture] \node (B1_top) {$B_1$};
    & \tikz[remember picture] \node (B2_top) {$B_2$};
    & \tikz[remember picture] \node (B3_top) {$B_3$};
    & \tikz[remember picture] \node (B4_top) {$B_4$};
    & \tikz[remember picture] \node (B5_top) {$B_5$};
    \\

    \cmidrule{2-7}
    \tikz[remember picture] \node (A1_left) {};
    & $A_1$
    & $1$
    & $0$
    & $0$
    & $0$
    & $2$
    & \tikz[remember picture] \node (A1_right) {};
    \\

    \tikz[remember picture] \node (A2_left) {};
    & $A_2$
    & $1$
    & $3$
    & $0$
    & $1$
    & $5$
    & \tikz[remember picture] \node (A2_right) {};
    \\

    \tikz[remember picture] \node (A3_left) {};
    & $A_3$
    & $2$
    & $0$
    & $1$
    & $3$
    & $0$
    & \tikz[remember picture] \node (A3_right) {};
    \\

    \tikz[remember picture] \node (A4_left) {};
    & $A_4$
    & $1$
    & $1$
    & $0$
    & $2$
    & $0$
    & \tikz[remember picture] \node (A4_right) {};
    \\

    \tikz[remember picture] \node (A5_left) {};
    & $A_5$
    & $0$
    & $0$
    & $0$
    & $0$
    & $0$
    & \tikz[remember picture] \node (A5_right) {};
    \\

    \multicolumn{2}{c}{}
    & \tikz[remember picture] \node (B1_bottom) {};
    & \tikz[remember picture] \node (B2_bottom) {};
    & \tikz[remember picture] \node (B3_bottom) {};
    & \tikz[remember picture] \node (B4_bottom) {};
    & \tikz[remember picture] \node (B5_bottom) {};
  \end{tabular}
  \begin{tikzpicture}[remember picture, overlay]
    \draw [blue!50] (B3_top.south) -- (B3_bottom.center);

    \draw [blue!50] (A3_left.center) -- (A3_right.center);
    \draw [blue!50] (A4_left.center) -- (A4_right.center);
    \draw [blue!50] (A5_left.center) -- (A5_right.center);
  \end{tikzpicture}
  $(+r^*) \leadsto$
  \begin{tabular}{cc|cccccc}
    &
    & \tikz[remember picture] \node (B1_top) {$B_1$};
    & \tikz[remember picture] \node (B2_top) {$B_2$};
    & \tikz[remember picture] \node (B3_top) {$B_3$};
    & \tikz[remember picture] \node (B4_top) {$B_4$};
    & \tikz[remember picture] \node (B5_top) {$B_5$};
    \\

    \cmidrule{2-7}
    \tikz[remember picture] \node (A1_left) {};
    & $A_1$
    & $1$
    & $0$
    & $0$
    & $0$
    & $2$
    & \tikz[remember picture] \node (A1_right) {};
    \\

    \tikz[remember picture] \node (A2_left) {};
    & $A_2$
    & $1$
    & $3$
    & $0$
    & $1$
    & $5$
    & \tikz[remember picture] \node (A2_right) {};
    \\

    \tikz[remember picture] \node (A3_left) {};
    & $A_3$
    & $2$
    & $0$
    & \underline{$3$}
    & $3$
    & $0$
    & \tikz[remember picture] \node (A3_right) {};
    \\

    \tikz[remember picture] \node (A4_left) {};
    & $A_4$
    & $1$
    & $1$
    & \underline{$2$}
    & $2$
    & $0$
    & \tikz[remember picture] \node (A4_right) {};
    \\

    \tikz[remember picture] \node (A5_left) {};
    & $A_5$
    & $0$
    & $0$
    & \underline{$2$}
    & $0$
    & $0$
    & \tikz[remember picture] \node (A5_right) {};
    \\

    \multicolumn{2}{c}{}
    & \tikz[remember picture] \node (B1_bottom) {};
    & \tikz[remember picture] \node (B2_bottom) {};
    & \tikz[remember picture] \node (B3_bottom) {};
    & \tikz[remember picture] \node (B4_bottom) {};
    & \tikz[remember picture] \node (B5_bottom) {};
  \end{tabular}
\end{center}

Und jetzt geht der Spaß wieder von vorne los. Wir müssen $d(R)$ bestimmen und
falls wir noch nicht genug Decklinien haben, aktualisieren wir und wiederholen
weiter, bis wir irgendwann eine optimale Zuordnung gefunden haben.

Bestimmen wir erneut $d(R)$ von unserem aktuellen Stand:
\begin{center}
  \begin{tabular}{cc|cccccc}
    &
    & \tikz[remember picture] \node (B1_top) {$B_1$};
    & \tikz[remember picture] \node (B2_top) {$B_2$};
    & \tikz[remember picture] \node (B3_top) {$B_3$};
    & \tikz[remember picture] \node (B4_top) {$B_4$};
    & \tikz[remember picture] \node (B5_top) {$B_5$};
    \\

    \cmidrule{2-7}
    \tikz[remember picture] \node (A1_left) {};
    & $A_1$
    & $1$
    & $0$
    & $0$
    & $0$
    & $2$
    & \tikz[remember picture] \node (A1_right) {};
    \\

    \tikz[remember picture] \node (A2_left) {};
    & $A_2$
    & $1$
    & $3$
    & $0$
    & $1$
    & $5$
    & \tikz[remember picture] \node (A2_right) {};
    \\

    \tikz[remember picture] \node (A3_left) {};
    & $A_3$
    & $2$
    & $0$
    & $3$
    & $3$
    & $0$
    & \tikz[remember picture] \node (A3_right) {};
    \\

    \tikz[remember picture] \node (A4_left) {};
    & $A_4$
    & $1$
    & $1$
    & $2$
    & $2$
    & $0$
    & \tikz[remember picture] \node (A4_right) {};
    \\

    \tikz[remember picture] \node (A5_left) {};
    & $A_5$
    & $0$
    & $0$
    & $2$
    & $0$
    & $0$
    & \tikz[remember picture] \node (A5_right) {};
    \\

    \multicolumn{2}{c}{}
    & \tikz[remember picture] \node (B1_bottom) {};
    & \tikz[remember picture] \node (B2_bottom) {};
    & \tikz[remember picture] \node (B3_bottom) {};
    & \tikz[remember picture] \node (B4_bottom) {};
    & \tikz[remember picture] \node (B5_bottom) {};
  \end{tabular}
\end{center}

\begin{enumerate}[(D1)]
  \item Wähle unabhängige Nullen aus (z.\,B. spaltenweise); diese sind
    Zuordnungen.
\end{enumerate}

\begin{center}
  \begin{tabular}{cc|cccccc}
    &
    & \tikz[remember picture] \node (B1_top) {$B_1$};
    & \tikz[remember picture] \node (B2_top) {$B_2$};
    & \tikz[remember picture] \node (B3_top) {$B_3$};
    & \tikz[remember picture] \node (B4_top) {$B_4$};
    & \tikz[remember picture] \node (B5_top) {$B_5$};
    \\

    \cmidrule{2-7}
    \tikz[remember picture] \node (A1_left) {};
    & $A_1$
    & $1$
    & $0$
    & $0$
    & \fbox{$0$}
    & $2$
    & \tikz[remember picture] \node (A1_right) {};
    \\

    \tikz[remember picture] \node (A2_left) {};
    & $A_2$
    & $1$
    & $3$
    & \fbox{$0$}
    & $1$
    & $5$
    & \tikz[remember picture] \node (A2_right) {};
    \\

    \tikz[remember picture] \node (A3_left) {};
    & $A_3$
    & $2$
    & \fbox{$0$}
    & $3$
    & $3$
    & $0$
    & \tikz[remember picture] \node (A3_right) {};
    \\

    \tikz[remember picture] \node (A4_left) {};
    & $A_4$
    & $1$
    & $1$
    & $2$
    & $2$
    & \fbox{$0$}
    & \tikz[remember picture] \node (A4_right) {};
    \\

    \tikz[remember picture] \node (A5_left) {};
    & $A_5$
    & \fbox{$0$}
    & $0$
    & $2$
    & $0$
    & $0$
    & \tikz[remember picture] \node (A5_right) {};
    \\

    \multicolumn{2}{c}{}
    & \tikz[remember picture] \node (B1_bottom) {};
    & \tikz[remember picture] \node (B2_bottom) {};
    & \tikz[remember picture] \node (B3_bottom) {};
    & \tikz[remember picture] \node (B4_bottom) {};
    & \tikz[remember picture] \node (B5_bottom) {};
  \end{tabular}
\end{center}

\begin{enumerate}[(D2)]
  \item Markiere Zeilen ohne Zuordnungen (Zeilen ohne ausgewählte Nullen).
\end{enumerate}

Es gibt keine Zeilen ohne Zuordnungen, also müssen wir nichts markieren.

\begin{enumerate}[(D3)]
  \item Markiere die noch nicht markierten Spalten, welche Nullen in den bereits
    markierten Zeilen haben.
\end{enumerate}

Da keine Zeilen markiert sind, müssen wir auch keine Spalten markieren.

\begin{enumerate}[(D4)]
  \item Markiere die noch nicht markierten Zeilen, welche Zuordnungen in den
    markierten Spalten haben.
\end{enumerate}
Da wir keine markierten Spalten haben, ändert sich auch hier nichts.
Damit sind wir auch schon am Ende von der Bestimmung von $d(R)$. Wir erhalten
\begin{align*}
  d(R) = \text{nichtmarkierte Zeilen} + \text{markierte Spalten} = 5 + 0 = 5
\end{align*}
Eingezeichnet sieht das wie folgt aus:
\begin{center}
  \begin{tabular}{cc|cccccc}
    &
    & \tikz[remember picture] \node (B1_top) {$B_1$};
    & \tikz[remember picture] \node (B2_top) {$B_2$};
    & \tikz[remember picture] \node (B3_top) {$B_3$};
    & \tikz[remember picture] \node (B4_top) {$B_4$};
    & \tikz[remember picture] \node (B5_top) {$B_5$};
    \\

    \cmidrule{2-7}
    \tikz[remember picture] \node (A1_left) {};
    & $A_1$
    & $1$
    & $0$
    & $0$
    & \fbox{$0$}
    & $2$
    & \tikz[remember picture] \node (A1_right) {};
    \\

    \tikz[remember picture] \node (A2_left) {};
    & $A_2$
    & $1$
    & $3$
    & \fbox{$0$}
    & $1$
    & $5$
    & \tikz[remember picture] \node (A2_right) {};
    \\

    \tikz[remember picture] \node (A3_left) {};
    & $A_3$
    & $2$
    & \fbox{$0$}
    & $3$
    & $3$
    & $0$
    & \tikz[remember picture] \node (A3_right) {};
    \\

    \tikz[remember picture] \node (A4_left) {};
    & $A_4$
    & $1$
    & $1$
    & $2$
    & $2$
    & \fbox{$0$}
    & \tikz[remember picture] \node (A4_right) {};
    \\

    \tikz[remember picture] \node (A5_left) {};
    & $A_5$
    & \fbox{$0$}
    & $0$
    & $2$
    & $0$
    & $0$
    & \tikz[remember picture] \node (A5_right) {};
    \\

    \multicolumn{2}{c}{}
    & \tikz[remember picture] \node (B1_bottom) {};
    & \tikz[remember picture] \node (B2_bottom) {};
    & \tikz[remember picture] \node (B3_bottom) {};
    & \tikz[remember picture] \node (B4_bottom) {};
    & \tikz[remember picture] \node (B5_bottom) {};
  \end{tabular}
  \begin{tikzpicture}[remember picture, overlay]
    \draw [blue!50] (A1_left.center) -- (A1_right.center);
    \draw [blue!50] (A2_left.center) -- (A2_right.center);
    \draw [blue!50] (A3_left.center) -- (A3_right.center);
    \draw [blue!50] (A4_left.center) -- (A4_right.center);
    \draw [blue!50] (A5_left.center) -- (A5_right.center);
  \end{tikzpicture}
\end{center}
Da $d(R) = 5 = 5$ gilt, sind wir fertig. Die Zuordnung der Bewerber auf die
Stellen lautet:
\begin{center}
  \begin{tabular}{cc|cccccc}
    &
    & \tikz[remember picture] \node (B1_top) {$B_1$};
    & \tikz[remember picture] \node (B2_top) {$B_2$};
    & \tikz[remember picture] \node (B3_top) {$B_3$};
    & \tikz[remember picture] \node (B4_top) {$B_4$};
    & \tikz[remember picture] \node (B5_top) {$B_5$};
    \\

    \cmidrule{2-7}
    \tikz[remember picture] \node (A1_left) {};
    & $A_1$
    & $0$
    & $0$
    & $0$
    & $1$
    & $0$
    & \tikz[remember picture] \node (A1_right) {};
    \\

    \tikz[remember picture] \node (A2_left) {};
    & $A_2$
    & $0$
    & $0$
    & $1$
    & $0$
    & $0$
    & \tikz[remember picture] \node (A2_right) {};
    \\

    \tikz[remember picture] \node (A3_left) {};
    & $A_3$
    & $0$
    & $1$
    & $0$
    & $0$
    & $0$
    & \tikz[remember picture] \node (A3_right) {};
    \\

    \tikz[remember picture] \node (A4_left) {};
    & $A_4$
    & $0$
    & $0$
    & $0$
    & $0$
    & $1$
    & \tikz[remember picture] \node (A4_right) {};
    \\

    \tikz[remember picture] \node (A5_left) {};
    & $A_5$
    & $1$
    & $0$
    & $0$
    & $0$
    & $0$
    & \tikz[remember picture] \node (A5_right) {};
    \\

    \multicolumn{2}{c}{}
    & \tikz[remember picture] \node (B1_bottom) {};
    & \tikz[remember picture] \node (B2_bottom) {};
    & \tikz[remember picture] \node (B3_bottom) {};
    & \tikz[remember picture] \node (B4_bottom) {};
    & \tikz[remember picture] \node (B5_bottom) {};
  \end{tabular}
\end{center}
Die Summe der Eignungspunkte ist damit $6+10+6+6 = 28$. Bewerber $B_1$ ging leer
aus.

\paragraph{Aufgabe 14}%
\label{par:aufgabe_14}

Eine Firma in Rostock will einer Firma in Bangkok einen Geldbetrag überweisen.
Für die Übermittlung des Geldes stehen verschiedene Wege zur Verfügung. Leider
vermindert sich die Summe auf jedem Teilstück des Übertragungsweges um einen
gewissen Prozentsatz (Bearbeitungsprovision) des jeweils verbliebenen Betrages.

Für die Abwicklung der Überweisung soll ein Weg gewählt werden, auf dem beim
Empfänger ein möglichst hoher Anteil der ursprünglich angewiesenen Summe
ankommt.

\begin{center}
  \begin{tikzpicture}[auto]
    \node [shape=rectangle, draw] (rostock) at (0,0) {Rostock};

    \node [shape=circle, draw] (a) at (3,2) {$a$};
    \node [shape=circle, draw] (b) at (3,0) {$b$};
    \node [shape=circle, draw] (c) at (3,-2) {$c$};

    \node [shape=circle, draw] (d) at (6,2) {$d$};
    \node [shape=circle, draw] (e) at (6,0) {$e$};
    \node [shape=circle, draw] (f) at (6,-2) {$f$};

    \node [shape=rectangle, draw] (bangkok) at (9,0) {Bangkok};

    \draw [-latex] (rostock) to [bend left] node {\SI{3}{\percent}} (a);
    \draw [-latex] (rostock) to node {\SI{8}{\percent}} (b);
    \draw [-latex] (rostock) to [bend right] node {\SI{1}{\percent}} (c);
    \draw [-latex] (a) to node [very near start, sloped] {\SI{3}{\percent}} (e);
    \draw [-latex] (b) to node [pos=0.3, sloped] {\SI{6}{\percent}} (d);
    \draw [-latex] (b) to node [very near start, sloped] {\SI{4}{\percent}} (f);
    \draw [-latex] (c) to node [near start, sloped] {\SI{5}{\percent}} (e);
    \draw [-latex] (d) to node {\SI{2}{\percent}} (bangkok);
    \draw [-latex] (e) to node [near start, above] {\SI{2}{\percent}} (b);
    \draw [-latex] (e) to node {\SI{10}{\percent}} (bangkok);
    \draw [-latex] (f) to node {\SI{4}{\percent}} (bangkok);
  \end{tikzpicture}
\end{center}

Formen Sie die Kantenbewertungen so um, dass durch die Aufgabenstellung ein
Kürzeste-Wege-Problem gelöst wird und lösen Sie das Überweisungsproblem im
obigen Bild mithilfe des Dijkstra-Algorithmus. Dabei sollen die auftretenden
Zahlen auf 3 Nachkommastellen kaufmännisch gerundet werden.

Zunächst müssen wir die Kantengewichte umformen. Dabei müssen wir folgendes
beachten: wenn wir den Betrag $x$ von Rostock nach $a$ überweisen und dabei
\SI{3}{\percent} Provision zahlen müssen, dann bleiben nach der Überweisung noch
\SI{97}{\percent} von $x$ übrig, also $\num{0,97} x$. Wenn dieser Betrag nach
$e$ überwiesen wird, dann werden von $\num{0,97} x$ die \SI{3}{\percent}
Provision abgezogen, sodass bei $e$ noch $\num{0,97} \cdot \num{0,97} x =
\num{0,941} x$ übrig bleiben (wie angegeben gerundet). Absolut gesehen werden
von Rostock über $a$ zu $e$ \quad $1-\num{0,941} = \SI{5,9}{\percent}$
abgezogen. Von $a$ zu $e$ werden dabei absolut $\SI{5,9}{\percent} -
\SI{3}{\percent} = \SI{2,9}{\percent}$ abgezogen. Dabei bedeutet
\textit{absolut}: absolut vom Betrag $x$. Wenn wir die Kantengewichte so
umrechnen, dann lassen sich die Prozente entlang eines Pfades addieren und wir
können mit dem Dijkstra-Algorithmus den kürzesten Pfad finden, auf dem die Summe
der Kantengewichte minimal ist.

Allgemein betrachtet ergibt sich folgendes: jede Kante vom Startknoten (Rostock)
ändert sich nicht. Entlang eines Pfades mit den Gewichten $p_1, p_2, \ldots,
p_n$ ergibt sich das Gewicht der letzten Kante aus
$(1-p_1) \cdot (1-p_2) \cdots (1-p_{n-1})
- (1-p_1) \cdot (1-p_2) \cdots (1-p_n)$.

Überprüfen wir das kurz: für die Kanten von Rostock nach $a, b$ und $c$ ändert
sich nichts. Für die Kante $ae$ ergibt sich ein Gewicht von
\begin{align*}
  (1-\num{0,03}) - (1-\num{0,03}) \cdot (1-\num{0,03})
  = \num{0,97} - \num{0,941}
  = \num{0,029}
\end{align*}
Soweit, so gut. Schauen wir uns $ce$ an:
\begin{align*}
  (1-\num{0,01}) - (1-\num{0,01}) \cdot (1-\num{0,05})
  = \num{0,99} - \num{0,941}
  = \num{0,049}
\end{align*}
An dem Zwischenergebnis $\num{0,99} - \num{0,941}$ sehen wir, dass wir richtig
liegen. Bei $c$ waren noch \SI{99}{\percent} übrig und bei $e$ sind noch
$\SI{99}{\percent} \cdot \SI{95}{\percent} = \SI{94,1}{\percent}$ übrig. Die
Differenz gibt an, wie viel auf der Kante abgezogen wurde.

Daraus entsteht folgender Graph:
\begin{center}
  \begin{tikzpicture}[auto]
    \node [shape=rectangle, draw] (rostock) at (0,0) {Rostock};

    \node [shape=circle, draw] (a) at (3,2) {$a$};
    \node [shape=circle, draw] (b) at (3,0) {$b$};
    \node [shape=circle, draw] (c) at (3,-2) {$c$};

    \node [shape=circle, draw] (d) at (6,2) {$d$};
    \node [shape=circle, draw] (e) at (6,0) {$e$};
    \node [shape=circle, draw] (f) at (6,-2) {$f$};

    \node [shape=rectangle, draw] (bangkok) at (9,0) {Bangkok};

    \draw [-latex] (rostock) to [bend left] node {\SI{3}{\percent}} (a);
    \draw [-latex] (rostock) to node {\SI{8}{\percent}} (b);
    \draw [-latex] (rostock) to [bend right] node {\SI{1}{\percent}} (c);
    \draw [-latex] (a) to node [very near start, sloped] {\SI{2,9}{\percent}} (e);
    \draw [-latex] (b) to node [pos=0.4, sloped] {\SI{5,5}{\percent}} (d);
    \draw [-latex] (b) to node [very near start, sloped] {\SI{3,7}{\percent}} (f);
    \draw [-latex] (c) to node [pos=0.4, sloped] {\SI{4,9}{\percent}} (e);
    \draw [-latex] (d) to node [sloped] {\SI{1,7}{\percent}} (bangkok);
    \draw [-latex] (e) to node [pos=0.4, above] {\SI{1,9}{\percent}} (b);
    \draw [-latex] (e) to node {\SI{9,4}{\percent}} (bangkok);
    \draw [-latex] (f) to node [sloped] {\SI{3,5}{\percent}} (bangkok);
  \end{tikzpicture}
\end{center}
Darauf können wir jetzt den Dijkstra-Algorithmus anwenden. Wir initialisieren
zunächst das Distanzarray $\dist$ und das Vorgängerarray $\vor$:
\begin{align*}
  \begin{array}{lcccccccc}
    v: & \text{Rostock} & a & b & c & d & e & f & \text{Bangkok}\\
    \midrule
    \dist[v]:
      & 0 & \infty & \infty & \infty & \infty & \infty & \infty & \infty\\
    \vor[v]:
      & \nil
  \end{array}
\end{align*}
Weiterhin ist $Q = V = \{ \text{Rostock}, a, b, c, d, e, f, \text{Bangkok}\}$.
Wir zählen die Iterationen der äußeren „while $Q \neq \emptyset$ do“ Schleife
auf:
\begin{enumerate}
  \item Der Knoten $u \in Q$ mit kleinstem $\dist[u]$ ist Rostock (schließlich
    ist $\dist$ für alle anderen $\infty$).

    $Q$ wird zu $Q \setminus \{ u \} = \{ a, b, c, d, e, f, \text{Bangkok}\}$.

    Die Menge $N^+(u) \cup Q$ ist $\{ a, b, c \}$.
    Wir iterieren über alle Elemente daraus:
    \begin{description}
      \item[$v := a$:] $\dist[u] + \ell(u, v) < d[v]$?
        \begin{align*}
          \dist[\text{Rostock}] + \ell(\text{Rostock}, a) & < d[a]\\
          0 + 3 & < \infty \quad \richtig
        \end{align*}
        Also müssen wir $\dist$ und $\vor$ aktualisieren:
        \begin{align*}
          \dist[a] & := \dist[\text{Rostock}] + \ell(\text{Rostock}, a)
            = 0 + 3 = 3\\
          \vor[a] & := \text{Rostock}
        \end{align*}
        Der Überblick:
        \begin{align*}
          \begin{array}{lcccccccc}
            v: & \text{Rostock} & a & b & c & d & e & f & \text{Bangkok}\\
            \midrule
            \dist[v]:
              & 0 & 3 & \infty & \infty & \infty & \infty & \infty & \infty\\
            \vor[v]:
              & \nil & \text{Rostock}
          \end{array}
        \end{align*}

      \item[$v := b$:] $\dist[u] + \ell(u, v) < d[v]$?
        \begin{align*}
          \dist[\text{Rostock}] + \ell(\text{Rostock}, b) & < d[b]\\
          0 + 8 & < \infty \quad \richtig
        \end{align*}
        Also müssen wir $\dist$ und $\vor$ aktualisieren:
        \begin{align*}
          \dist[b] & := \dist[\text{Rostock}] + \ell(\text{Rostock}, b)
            = 0 + 8 = 8\\
          \vor[b] & := \text{Rostock}
        \end{align*}
        Der Überblick:
        \begin{align*}
          \begin{array}{lcccccccc}
            v: & \text{Rostock} & a & b & c & d & e & f & \text{Bangkok}\\
            \midrule
            \dist[v]:
              & 0 & 3 & 8 & \infty & \infty & \infty & \infty & \infty\\
            \vor[v]:
              & \nil & \text{Rostock} & \text{Rostock}
          \end{array}
        \end{align*}

      \item[$v := c$:] $\dist[u] + \ell(u, v) < d[v]$?
        \begin{align*}
          \dist[\text{Rostock}] + \ell(\text{Rostock}, c) & < d[c]\\
          0 + 1 & < \infty \quad \richtig
        \end{align*}
        Also müssen wir $\dist$ und $\vor$ aktualisieren:
        \begin{align*}
          \dist[c] & := \dist[\text{Rostock}] + \ell(\text{Rostock}, c)
            = 0 + 1 = 1\\
          \vor[c] & := \text{Rostock}
        \end{align*}
        Der Überblick:
        \begin{align*}
          \begin{array}{lcccccccc}
            v: & \text{Rostock} & a & b & c & d & e & f & \text{Bangkok}\\
            \midrule
            \dist[v]:
              & 0 & 3 & 8 & 1 & \infty & \infty & \infty & \infty\\
            \vor[v]:
              & \nil & \text{Rostock} & \text{Rostock} & \text{Rostock}
          \end{array}
        \end{align*}
    \end{description}

  \item Damit ist die erste Iteration durch und wir starten gleich in die
    nächste.  Wir suchen wieder den Knoten $u \in Q$ mit dem kleinesten
    $\dist[u]$. Das ist $c$ mit $\dist[c] = 1$. Wir entfernen $c$ aus $Q$:
    \begin{align*}
      Q & := Q \setminus \{ c \} = \{ a, b, d, e, f, \text{Bangkok}\}
    \end{align*}
    Und iterieren über alle Nachbarn von $c$, die zudem noch in $Q$ sind. Diese
    Menge $N^+(u) \cup Q$ ist nur $\{ e \}$:
    \begin{description}
      \item[$v := e$:] $\dist[u] + \ell(u, v) < d[v]$?
        \begin{align*}
          \dist[c] + \ell(c, e) & < d[e]\\
          1 + \num{4,9} & < \infty \quad \richtig
        \end{align*}
        Also müssen wir $\dist$ und $\vor$ aktualisieren:
        \begin{align*}
          \dist[e] & := \dist[c] + \ell(c, e)
            = 1 + \num{4,9} = \num{5,9}\\
          \vor[e] & := c
        \end{align*}
        Der Überblick:
        \begin{align*}
          \begin{array}{lcccccccc}
            v: & \text{Rostock} & a & b & c & d & e & f & \text{Bangkok}\\
            \midrule
            \dist[v]:
              & 0 & 3 & 8 & 1 & \infty & \num{5,9} & \infty & \infty\\
            \vor[v]:
              & \nil & \text{Rostock} & \text{Rostock} & \text{Rostock}
              & & c
          \end{array}
        \end{align*}
    \end{description}

  \item Zweite Iteration durch, die dritte folgt.
    Das $u \in Q$ mit kleinstem $\dist[u]$ ist $a$.
    $Q$ wird zu $\{ b, d, e, f, \text{Bangkok}\}$.
    Wir iterieren wieder über $N^+(u) \cup Q = \{ e \}$:
    \begin{description}
      \item[$v := e$:] $\dist[u] + \ell(u, v) < d[v]$?
        \begin{align*}
          \dist[a] + \ell(a, e) & < d[e]\\
          3 + \num{4,9} & < \num{5,9} \quad \falsch
        \end{align*}
        Also muss nichts aktualisiert werden.
    \end{description}

  \item Das $u \in Q$ mit kleinstem $\dist[u]$ ist $e$.
    $Q$ wird zu $\{ b, d, f, \text{Bangkok}\}$.
    Wir iterieren über $N^+(u) \cup Q = \{ b, \text{Bangkok} \}$:
    \begin{description}
      \item[$v := b$:] $\dist[u] + \ell(u, v) < d[v]$?
        \begin{align*}
          \dist[e] + \ell(e, b) & < d[b]\\
          \num{5,9} + \num{1,9} & < \num{8} \quad \richtig
        \end{align*}
        Also müssen wir $\dist$ und $\vor$ aktualisieren:
        \begin{align*}
          \dist[b] & := \dist[e] + \ell(e, b)
            = \num{5,9} + \num{1,9} = \num{7,8}\\
          \vor[b] & := e
        \end{align*}
        Der Überblick:
        \begin{align*}
          \begin{array}{lcccccccc}
            v: & \text{Rostock} & a & b & c & d & e & f & \text{Bangkok}\\
            \midrule
            \dist[v]:
              & 0 & 3 & \num{7,8} & 1 & \infty & \num{5,9} & \infty & \infty\\
            \vor[v]:
              & \nil & \text{Rostock} & e & \text{Rostock} & & c
          \end{array}
        \end{align*}

      \item[$v := \textnormal{Bangkok}$:] $\dist[u] + \ell(u, v) < d[v]$?
        \begin{align*}
          \dist[e] + \ell(e, \text{Bangkok}) & < d[\text{Bangkok}]\\
          \num{5,9} + \num{9,4} & < \infty \quad \richtig
        \end{align*}
        Also müssen wir $\dist$ und $\vor$ aktualisieren:
        \begin{align*}
          \dist[\text{Bangkok}] & := \dist[e] + \ell(e, \text{Bangkok})
            = \num{5,9} + \num{9,4}= \num{15,3}\\
          \vor[\text{Bangkok}] & := e
        \end{align*}
        Der Überblick:
        \begin{align*}
          \begin{array}{lcccccccc}
            v: & \text{Rostock} & a & b & c & d & e & f & \text{Bangkok}\\
            \midrule
            \dist[v]:
              & 0 & 3 & \num{7,8} & 1 & \infty & \num{5,9} & \infty & \num{15,3}\\
            \vor[v]:
              & \nil & \text{Rostock} & e & \text{Rostock} & & c & & e
          \end{array}
        \end{align*}
    \end{description}

  \item Das $u \in Q$ mit kleinstem $\dist[u]$ ist $b$.
    $Q$ wird zu $\{ d, f, \text{Bangkok}\}$.
    Wir iterieren über $N^+(u) \cup Q = \{ d, f \}$:

    \begin{description}
      \item[$v := d$:] $\dist[u] + \ell(u, v) < d[v]$?
        \begin{align*}
          \dist[b] + \ell(b, d) & < d[d]\\
          \num{7,8} + \num{5,5} & < \infty \quad \richtig
        \end{align*}
        Also müssen wir $\dist$ und $\vor$ aktualisieren:
        \begin{align*}
          \dist[d] & := \dist[b] + \ell(b, d)
            = \num{7,8} + \num{5,5}= \num{13,3}\\
          \vor[d] & := b
        \end{align*}
        Der Überblick:
        \begin{align*}
          \begin{array}{lcccccccc}
            v: & \text{Rostock} & a & b & c & d & e & f & \text{Bangkok}\\
            \midrule
            \dist[v]:
              & 0 & 3 & \num{6,8} & 1 & \num{13,3} & \num{5,9} & \infty & \num{14,3}\\
            \vor[v]:
              & \nil & \text{Rostock} & e & \text{Rostock} & b & c & & e
          \end{array}
        \end{align*}

      \item[$v := f$:] $\dist[u] + \ell(u, v) < d[v]$?
        \begin{align*}
          \dist[b] + \ell(b, f) & < d[f]\\
          \num{7,8} + \num{3,7} & < \infty \quad \richtig
        \end{align*}
        Also müssen wir $\dist$ und $\vor$ aktualisieren:
        \begin{align*}
          \dist[f] & := \dist[b] + \ell(b, f)
            = \num{7,8} + \num{3,7}= \num{11,5}\\
          \vor[f] & := b
        \end{align*}
        Der Überblick:
        \begin{align*}
          \begin{array}{lcccccccc}
            v: & \text{Rostock} & a & b & c & d & e & f & \text{Bangkok}\\
            \midrule
            \dist[v]:
              & 0 & 3 & \num{7,8} & 1 & \num{13,3} & \num{5,9} & \num{11,5} & \num{15,3}\\
            \vor[v]:
              & \nil & \text{Rostock} & e & \text{Rostock} & b & c & b & e
          \end{array}
        \end{align*}
    \end{description}

  \item Das $u \in Q$ mit kleinstem $\dist[u]$ ist $f$.
    $Q$ wird zu $\{ d, \text{Bangkok}\}$.
    Wir iterieren über $N^+(u) \cup Q = \{ \text{Bangkok} \}$:

    \begin{description}
      \item[$v := \textnormal{Bangkok}$:] $\dist[u] + \ell(u, v) < d[v]$?
        \begin{align*}
          \dist[f] + \ell(f, \text{Bangkok}) & < d[\text{Bangkok}]\\
          \num{11,5} + \num{3,5} & < \num{15,3} \quad \richtig
        \end{align*}
        Also müssen wir $\dist$ und $\vor$ aktualisieren:
        \begin{align*}
          \dist[\text{Bangkok}] & := \dist[f] + \ell(f, \text{Bangkok})
            = \num{11,5} + \num{3,5}= \num{15}\\
          \vor[\text{Bangkok}] & := f
        \end{align*}
        Der Überblick:
        \begin{align*}
          \begin{array}{lcccccccc}
            v: & \text{Rostock} & a & b & c & d & e & f & \text{Bangkok}\\
            \midrule
            \dist[v]:
              & 0 & 3 & \num{7,8} & 1 & \num{13,3} & \num{5,9} & \num{11,5} & \num{15}\\
            \vor[v]:
              & \nil & \text{Rostock} & e & \text{Rostock} & b & c & b & f
          \end{array}
        \end{align*}
    \end{description}

  \item Das $u \in Q$ mit kleinstem $\dist[u]$ ist $d$.
    $Q$ wird zu $\{ \text{Bangkok}\}$.
    Wir iterieren über $N^+(u) \cup Q = \{ \text{Bangkok} \}$:

    \begin{description}
      \item[$v := \textnormal{Bangkok}$:] $\dist[u] + \ell(u, v) < d[v]$?
        \begin{align*}
          \dist[d] + \ell(d, \text{Bangkok}) & < d[\text{Bangkok}]\\
          \num{13,3} + \num{1,7} & < \num{15} \quad \falsch
        \end{align*}
        Also müssen wir nichts aktualisieren.
    \end{description}

  \item Bleibt zuletzt $u := \text{Bangkok}$.
    $Q$ wird zu $\emptyset$.
    Die Menge $N^+(u) \cup Q$ ist leer, also brauchen wir nichts machen.
\end{enumerate}

Ausgehend vom letzten Überblick
\begin{align*}
  \begin{array}{lcccccccc}
    v: & \text{Rostock} & a & b & c & d & e & f & \text{Bangkok}\\
    \midrule
    \dist[v]:
      & 0 & 3 & \num{7,8} & 1 & \num{13,3} & \num{5,9} & \num{11,5} & \num{15}\\
    \vor[v]:
      & \nil & \text{Rostock} & e & \text{Rostock} & b & c & b & f
  \end{array}
\end{align*}
ist der kürzeste Weg von Rostock nach Bangkok:
\begin{align*}
  \text{Rostock}
    \rightarrow c
    \rightarrow e
    \rightarrow b
    \rightarrow f
    \rightarrow \text{Bangkok}
\end{align*}
Dieser Weg hat Kosten von $15$, also muss man mindestens \SI{15}{\percent}
Bearbeitungsprovision bezahlen.

\paragraph{Aufgabe 15}%
\label{par:aufgabe_15}

Sie möchten $100$ US Dollar für möglichst wenig Euro erwerben. Zum einen können
Sie direkt den Preis für einen Dollar in Euro erfragen und dann den Betrag für
100 Dollar bezahlen. Oder Sie können beispielsweise zuerst Dollar für Währung
$x$ erstehen, anschließend $x$ für $y$ und dann erst $y$ für Euro kaufen. Dabei
werden die $100$ US Dollar mit dem jeweiligen Wechselkurs multipliziert bis man
am Ende Euros erhält. Natürlich soll ein möglichst günstiger Kaufweg gefunden
werden.

Einige Devisenkurse vom 21. Juni 2018 (onvista.de):
\begin{center}
  \begin{tabular}{lrrrrrr}
    \toprule
             & EUR          & USD          & RUB           & JPY            & GBP          & CHF\\
    \midrule
    EUR      & —            & \num{1,1546} & \num{73,5960} & \num{127,6220} & \num{0,8734} & \num{1,1496}\\
    USD      & \num{0,8661} & —            & \num{63,7773} & \num{110,5340} & \num{0,7565} & \num{0,9954}\\
    RUB      & \num{0,0135} & \num{0,0156} & —             & \num{1,7334}   & \num{0,0118} & \num{0,0156}\\
    JPY      & \num{0,0078} & \num{0,0090} & \num{0,5769}  & —              & \num{0,0068} & \num{0,0090}\\
    GBP      & \num{1,1448} & \num{1,3219} & \num{84,2200} & \num{146,1130} & —            & \num{1,3162}\\
    CHF      & \num{0,8692} & \num{1,0042} & \num{63,9690} & \num{111,0060} & \num{0,7592} & —\\
    \bottomrule
  \end{tabular}
\end{center}
Begründen Sie, dass hier ein Kürzeste-Wege-Problem vorliegt und welches
Lösungsverfahren Sie hierfür verwenden würden.

Beginnen wir mit einem Rechenbeispiel. Wenn man direkt $100$ US Dollar für Euro
haben möchte, dann gilt die Gleichung:
\begin{align*}
  \SI{100}{\USD} & = \SI{86,61}{\EUR}
\end{align*}

Schauen wir mal, was passiert, wenn wir einen Umweg gehen, so wie er in der
Aufgabenstellung beschrieben ist (wir nehmen RUB für $x$ und JPY für $y$).
\begin{align*}
  \SI{100}{\USD} &
    = 100 \cdot \SI{63,7773}{\RUB}
    = 100 \cdot \num{63,7773} \cdot \SI{1,7334}{\JPY}
    = 100 \cdot \num{63,7773} \cdot \num{1,7334} \cdot \SI{0,0078}{\EUR}
    = \SI{86,23}{\EUR}
\end{align*}
Das wäre günstiger, als Euros direkt für US Dollar zu tauschen.

Entlang des Pfades USD $\rightarrow$ RUB $\rightarrow$ JPY $\rightarrow$ EUR
werden die Umrechnungskurse multipliziert.

\end{document}
