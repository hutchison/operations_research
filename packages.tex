% _das_ Mathepaket schlechthin:
\usepackage[
  %% Nummerierung von Gleichungen links:
  leqno,
  %% Ausgabe von Gleichungen linksbündig:
  fleqn,
]{mathtools}
% und dazu noch ein paar Mathesymbole und so:
% (muss vor mathspec geladen werden)
\usepackage{amsmath, amssymb}

\usepackage[]{parskip}

% chemische Formeln
\usepackage{mhchem}

\usepackage{ifxetex}
\ifxetex
  % Um auch schöne Schriftarten auswählen zu können:
  \usepackage[MnSymbol]{mathspec}

  % Wir wollen, dass alle unsere Schriften für TeX und einander angepasst sind:
  \defaultfontfeatures{Mapping=tex-text, Scale=MatchLowercase}
  % Die Hauptschriftart:
  \setmainfont[]{Minion Pro}
  % Die Matheschriftart:
  \setmathfont(Digits,Latin,Greek)[
    Numbers={Lining, Proportional}
  ]{Minion Pro}
  \setmathrm{Minion Pro}
  % Die Schriftart für serifenlose Texte (z.B. Überschriften):
  \setallsansfonts[]{Myriad Pro}
  % Und die Schriftart für nichtproportionale Texte:
  \setallmonofonts[]{Fira Mono}
\fi

% Deutsche Sprache bei Silbentrennung und Datum:
\usepackage[ngerman]{babel}

% St. Mary Road, liefert Symbole für theoretische Informatik (z.B. \lightning):
%\usepackage{stmaryrd}

% nutze die volle Seite als Satzspiegel:
\usepackage[
  % Randbreite sei 1cm (sonst ist sie 1in):
  cm,
  % Kopf- und Fußzeile werden miteinbezogen:
  headings
]{fullpage}
% verbesserte Tabellen
% bietet u.a. die Spaltenmöglichkeit 'm{width}' = zentrierte Spalte mit fester
% Breite
\usepackage{array}
% kann komplexe Linien in Tabellen ziehen:
%\usepackage{hhline}
% mehrseitige Tabellen:
%\usepackage{longtable}
% Tabellen mit gedehnten Spalten:
\usepackage{tabularx}
% Pimpt enumerate auf (optionales Argument liefert Nummerierung):
\usepackage{enumerate}
% Kann descriptions auf die selbe Höhe bringen:
%\usepackage{enumitem}
% Liefert Hyperlinks (\hyperref, \url, \href}
\usepackage{hyperref}
\hypersetup{%
  colorlinks=true,
  linkcolor=black,
  urlcolor=blue,
}
\usepackage{cleveref}
% Farben (wie bei TikZ):
\usepackage[dvipsnames]{xcolor}
\usepackage[]{pifont}
% Ändert den Zeilenabstand:
\usepackage[
  % nur eine Möglichkeit auswählen:
  singlespacing
  %onehalfspacing
  %doublespacing
]{setspace}
% Schönere Tabellen
% dazu gibt's neue Kommandos:
% - \toprule[(Dicke)], \midrule[(Dicke)], \bottomrule[(Dicke)]
% - \addlinespace: Extrahöhe zwischen Zeilen
\usepackage{booktabs}
% Schöne numerische Zitierungen:
%\usepackage{cite}
% Verbessert den Satz von Abbildungsüberschriften:
%\usepackage{caption}
% Ermöglicht durch \begin{linenumbers} Zeilennummern anzuzeigen:
%\usepackage{lineno}
% Ermöglicht Zugriff auf die letzte Seite (z.B. \pageref{LastPage}):
\usepackage{lastpage}
% Quelltext schön setzen:
%\usepackage{listings}
% Logische Beweise:
%\usepackage{bussproofs}
% Unterstreichungen (\uline, \uuline, \sout: durchgestrichen, \uwave):
%\usepackage{ulem}
% Kann alle möglichen Maße ändern
% will man Querformat, dann:
%\usepackage[landscape]{geometry}
% bietet gestrichelte vert. Linien in Tabellen (':')
%\usepackage{arydshln}
% Quelltext schön setzen:
% (verlangt "xelatex -shell-escape"!)
%\usepackage{minted}
% um Bilder einzubinden:
%\usepackage{graphicx}
% um in Tabellen eine Zelle über mehrere Zeilen laufen zu lassen:
%\usepackage{multirow}
% SI-Einheiten mittels \si{}:
\usepackage[mode=text]{siunitx}
\sisetup{%
  output-decimal-marker={,},
  %per-mode=fraction,
  %exponent-product=\cdot,
}
%\DeclareSIUnit\cal{cal}
%\DeclareSIUnit\diopter{dpt}
%\DeclareSIUnit\fahrenheit{F}
% nette Brüche mittels \sfrac{}{}:
\usepackage{xfrac}

% Coole Zeichnungen:
\usepackage{tikz}
\usetikzlibrary{%
  %backgrounds,
  %mindmap,
  %shapes.geometric,
  %shapes.symbols,
  %shapes.misc,
  %shapes.multipart,
  %positioning,
  %fit,
  calc,
  arrows,
  %automata,
  %trees,
  %decorations.pathreplacing,
  %circuits.ee.IEC,
}

% eigens gebaute Lochmarken:
%\usepackage{eso-pic}
%\AddToShipoutPicture*{
  %\put(\LenToUnit{0mm},\LenToUnit{228.5mm})
    %{\rule{\LenToUnit{20pt}}{\LenToUnit{0.5pt}}}
  %\put(\LenToUnit{0mm},\LenToUnit{68.5mm})
    %{\rule{\LenToUnit{20pt}}{\LenToUnit{0.5pt}}}
%}

% Definitionen und Sätze:
\usepackage[]{amsthm}

\usepackage{wrapfig}

\newtheoremstyle{bonny}% schottisch für „ansehnlich“
  {9pt}% measure of space to leave above the theorem. E.g.: 3pt
  {6pt}% measure of space to leave below the theorem. E.g.: 3pt
  {}% name of font to use in the body of the theorem
  {}% measure of space to indent
  {\bfseries}% name of head font
  {\smallskip}% punctuation between head and body
  {\newline}% space after theorem head; " " = normal interword space
  {}% Manually specify head

\theoremstyle{bonny}

\newtheorem{definition}{Definition}
\newtheorem{gesetz}{Gesetz}
\newtheorem{satz}{Satz}
\newtheorem{beispiel}{Beispiel}
\newtheorem{bemerkung}{Bemerkung}

% coole Kopf- und Fußzeilen:
\usepackage{fancyhdr}
% Seitenstil ist natürlich fancy:
\pagestyle{fancy}
% alle Felder löschen:
\fancyhf{}
% Veranstaltung:
% Linie oben/unten:
\renewcommand{\headrulewidth}{0.0pt}
\renewcommand{\footrulewidth}{0.0pt}

\newcommand{\cmark}{\ding{51}}%
\newcommand{\xmark}{\ding{55}}%
\newcommand{\rot}[1]{\textcolor{BrickRed}{#1}}
\newcommand{\gruen}[1]{\textcolor{ForestGreen}{#1}}
\newcommand{\richtig}{\gruen{\text{\cmark}}}
\newcommand{\falsch}{\rot{\text{\xmark}}}
